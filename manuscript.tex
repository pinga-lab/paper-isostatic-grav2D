%\documentclass[paper,twocolumn,twoside]{geophysics}
%\documentclass[manuscript,revised]{geophysics}
\documentclass[manuscript]{geophysics}

% An example of defining macros
\newcommand{\rs}[1]{\mathstrut\mbox{\scriptsize\rm #1}}
\newcommand{\rr}[1]{\mbox{\rm #1}}

% Extra packages
\usepackage{amsmath}
\usepackage{bm}
\usepackage[pdftex,colorlinks=true]{hyperref}
\hypersetup{
	allcolors=black,
}
\usepackage{lipsum}

\renewcommand{\figdir}{figures} % figure directory

\begin{document}

\title{2D gravity inversion with isostatic constraint applied to passive rifted margins}

\renewcommand{\thefootnote}{\fnsymbol{footnote}} 

\ms{GEO-XXXX} % manuscript number

\address{
\footnotemark[1]Observat\'{o}rio Nacional, \\
Department of Geophysics, \\
Rio de Janeiro, Brazil}
\author{B. Marcela S. Bastos\footnotemark[1] and Vanderlei C. Oliveira Jr\footnotemark[1]}

\footer{vanderlei@on.br}
\lefthead{Bastos and Oliveira Jr.}
\righthead{2D gravity inversion for passive rifted margins}

\maketitle

\begin{abstract}

\lipsum[1]

\end{abstract}

\section{Introduction}

Several methods have been proposed for using gravity and/or magnetic data to estimate
the basement relief under sedimentary layers or the Moho.

These methods usually presume that the basement and Moho are interfaces that oscillates 
around a known reference depth and separate two layers having constant physical 
properties (density or magnetization).

In the case of basement, these two layers are the sedimentary package and the upper
crust. In the case of Moho, the two layer are the lower crust and upper mantle.



For different combinations of density values and reference depths, it is possible
to find different interfaces producing the same gravity data.

To deal with this inherent ambiguity \citep{roy1962, skeels1962} and obtain
meaningful solutions, the interpreter must use a priori information 
obtained from seismic data and/or boreholes, for example, in order to constrain
the range of possible models.


One of the first automated methods for estimating the interface separating
two homogeneous layers was presented by \citet{bott1960}, in space domain.

Since then, several other approaches have been developed in space domain
\citep{tanner1967, cordell-henderson1968, dyrelius-vogel1972, pedersen1977,
richardson-macinnes1989, barbosa-etal1997, barbosa-etal1999, barbosa-etal1999b,
silva-etal2006, chakravarthi-sundararajan2007, martins-etal2010, silva-etal2010,
camacho-etal2011, lima-etal2011, martins-etal2011, barnes-barraud2012, silva-etal2014,
santos-etal2015, silva-santos2017} and also in the Fourier domain 
\citep{oldenburg1974, granser1987, reamer-ferguson1989, guspi1993, 
braitenberg-etal1997, braitenberg-zadro1999}.

Gravity gradient data \citep{barnes-barraud2012}.

Methods in Fourier domain use Parker's formula \citep{parker1973}.

Moho from satellite data by using the Vening Meinesz-Moritz approach \citep{sjoberg2009, bagherbandi-eshagh2012} 

Satellite data \citep{reguzzoni-etal2013}.



\citep{sampietro2015} - pegar as refs citadas no trabalho "GOCE data to in-
fer the Moho depth or the density contrast between crust and upper
mantle at a regional/continental scale with extremely high accu-
racy (Braitenberg et al. 2010, 2011; Mariani et al. 2013; Van der Meijde et al. 2013; Sampietro et al.
2014; Bouman et al. 2015; Van der Meijde et al. 2015)"

Moho from satelitte data using tesseroids \citep{uieda-barbosa2017}.

\citep{salem-etal2014}

Magnetic data \citep{pilkington-crossley1986, pilkington-crossley1986b}

Joint inversion of gravity and magnetic data \citep{gallardo-etal2003, gallardo-etal2005, pilkington2006}




\section{Methodology}


\subsection{Forward problem}


Let $\mathbf{d}^{o}$ be the observed data vector, whose $i$-th element $d^{o}_{i}$, 
$i = 1, \dots, N$, represent the observed gravity disturbance at the point 
$(x_{i}, y_{i}, z_{i})$, on a profile located over a rifted passive margin. The
coordinates are referred to a topocentric Cartesian system, with $z$ axis pointing
down, $y$-axis along the profile and $x$-axis perpendicular to the profile. 
We assume that the observed gravity disturbance is produced by an anomalous 
mass distribution defined as the difference between the actual mass distribution
in the subsurface, which is schematically represented in Figure \ref{fig:mass-distribution},
and a reference mass distribution (Figure \ref{fig:reference-mass-distribution}).
In doing it, we implicitly assume that Figure \ref{fig:reference-mass-distribution}
represents the outer layers of a global mass distribution producing the
normal gravity field. 

The anomalous mass distribution producing the observed data
is approximated by an interpretation model (Figure \ref{fig:interpretation-model}) 
formed by $N$ adjacent columns. For convenience, we presume that the observed data are
regularly spaced, so that there is one observation at the centre of the top of each
column forming the interpretation model.The $i$-th column is formed by four vertically
adjacent 
layers, , which in turn are composed of vertically adjacent prisms having infinite 
length along the $x$-axis. The first and shallowest layer represents the water layer, 
is formed by a single prism, has thickness $t^{w}_{i}$ and a constant density contrast 
$\Delta \rho^{w} = \rho^{w} - \rho^{r}$, where $\rho^{w}$ and $\rho^{r}$ 
represents, respectively, the densities of water and the reference mass 
distribution (Figure \ref{fig:reference-mass-distribution}) at the same point.
The third layer represents the crust, it is also formed by a single prism,
has thickness $t^{c}_{i}$ and density contrast 
$\Delta \rho^{c}_{i} = \rho^{c} - \rho^{r}$, 
with $\rho^{c}$ being the crust density. For simplicity, we presume that the crust
density $\rho^{c}_{i}$ may be equal to $\rho^{cc}$, for $y_{i} \le y_{COT}$, which
represents continental crust,
or equal to $\rho^{oc}$, for $y_{i} > y_{COT}$, which represents oceanic crust.
The crust density depends on the position of the $i$-th column with respect to
$y_{COT}$, which defines an abrupt Crust-Ocean Transition (COT). Consequently, the
crust may have two possible density contrasts: 
$\Delta \rho^{c}_{i} = \rho^{cc} - \rho{r}$ or 
$\Delta \rho^{c}_{i} = \rho^{oc} - \rho^{r}$. The top of this layer defines the 
basement relief and its bottom the relief of the Moho. The fourth layer represents the
mantle, it is divided into two parts, each one formed by a single prism having the same
density $\rho^{m}$ and, consequently, the same density contrast 
$\Delta \rho^{m} = \rho^{m} - \rho^{r}$. The shallowest
portion of this layer has thickness $t^{m}_{i}$. Its top and bottom define,
respectively, the depths of Moho and the planar isostatic compensation layer $S_{0}$.
The deepest portion of the fourth layer has thickness $\Delta S_{0}$, top at the
surface $S_{0}$ and bottom at the planar surface 
$S_{0} + \Delta S_{0}$, which defines the Moho in the reference mass distribution model 
(Figure \ref{fig:reference-mass-distribution}). Finally, the second layer forming the
$t$-th column of the interpretation model is defined by the interpreter, according to
the geological environment to be studied and the a priori information availability. 
As a general rule, this layer can be defined by a set of $Q$ vertically adjacent
prisms, each one with thickness $t^{q}_{i}$, density $\rho^{q}$ and density contrast
$\Delta \rho^{q} = \rho^{q} - \rho^{r}$, $q = 1, \dots Q$.

Given the density contrasts, the COT position $y_{COT}$, the isostatic compensation
surface $S_{0}$, the thickness of the water layer and of the $Q-1$ prisms forming the
shallowest portion of the second layer, it is possible to describe the interpretation
model in terms of an $M \times 1$ parameter vector $\mathbf{p}$, $M = 2N + 1$, defined
as follows:
\begin{equation}
\mathbf{p} = \begin{bmatrix}
\mathbf{t}^{Q} \\
\mathbf{t}^{m} \\
\Delta S_{0}
\end{bmatrix} \: ,
\label{eq:parameter-vector}
\end{equation}
where $\mathbf{t}^{Q}$ and $\mathbf{t}^{m}$ are $N \times 1$ vectors whose $i$-th
elements $t^{Q}_{i}$ and $t^{m}_{i}$ represent, respectively, the thickness of the
prism forming the deepest portion of the second layer and the thickness of the prism
forming the shallowest portion of the fourth layer of the interpretation model.
In this case, the gravity disturbance produced by the interpretation model (the
predicted gravity disturbance) at the position $(x_{i}, y_{i}, z_{i})$ can be written
as the sum of the vertical component of the gravitational attraction exerted by the $L$
prisms forming the interpretation model as follows:
\begin{equation}
d_{i}(\mathbf{p}) = k_{g} \, G \, \sum_{j = 1}^{L} f_{ij}(\mathbf{p}) \: ,
\label{eq:ith-predicted-data}
\end{equation}
where $f_{ij}(\mathbf{p})$ represents an integral over the volume of the $j$-th 
prism. Here, these volume integrals are computed with the expressions proposed 
by \citet{nagy-etal2000}, by using the open-source Python package 
\textit{Fatiando a Terra} \citep{uieda-etal2013}.


\subsection{Inverse problem}


Let $\mathbf{d}(\mathbf{p})$ be the predicted data vector, whose $i$-th element
$d_{i}(\mathbf{p})$ is defined by Equation \ref{eq:ith-predicted-data}. Estimating the
particular parameter vector $\mathbf{p} = \hat{\mathbf{p}}$ producing a predicted data
vector $\mathbf{d}(\mathbf{p})$ as close as possible to the observed data vector 
$\mathbf{d}^{o}$ can be formulated as the problem of minimizing the goal function
\begin{equation}
\Gamma (\mathbf{p}) = \Phi(\mathbf{p}) + \mu \sum_{k = 0}^{3} \alpha_{k}
\Psi_{k}(\mathbf{p}) \: ,
\label{eq:goal-function}
\end{equation}
subject to all elements of $\hat{\mathbf{p}}$ be positive. In Equation
\ref{eq:goal-function}, $\mu$ represents the regularizing parameter, $\Phi(\mathbf{p})$
represents the misfit function given by
\begin{equation}
\Phi(\mathbf{p}) = \frac{1}{N} \| \mathbf{d}^{o} - \mathbf{d}(\mathbf{p}) \|_{2}^{2} 
\: , 
\label{eq:misfit-function}
\end{equation}
where $\| \cdot \|_{2}^{2}$ represents the squared Euclidean norm, $\alpha_{k}$
represent the weights assigned to the regularizing functions $\Psi_{k}(\mathbf{p})$,
with define the constraints on the parameters to be estimated, $k = 0, 1, 2, 3$.

\subsection{Airy constraint}

Consider that the interpretation model is in isostatic equilibrium according to the
Airy model \citep{turcotte-schubert2002, hofmann-wellenhof-moritz2005, lowrie2007}.
In this case, the pressure (or lithostatic stress) exerted by the model is constant 
on the isostatic compensation surface $S_{0}$. The pressure per unit area 
exerted by the $i$-th column of the model on $S_{0}$, divided by gravity, 
is given by:
\begin{equation}
t^{w}_{i} \rho^{w} + t^{1}_{i} \rho^{1}_{i} + \dots + t^{Q}_{i} \rho^{Q}_{i} + 
t^{c}_{i} \rho^{c}_{i} + t^{m}_{i} \rho^{m} = \sigma_{0} \: ,
\label{eq:lithostatic-stress-densities}
\end{equation}
where $\sigma_{0}$ is an arbitrary positive constant. Rearranging terms in Equation
\ref{eq:lithostatic-stress-densities} and using the relation
\begin{equation}
S_{0} = t^{w}_{i} + t^{1}_{i} + \dots + t^{Q}_{i} + t^{c}_{i} + t^{m}_{i} \: ,
\label{eq:S0}
\end{equation}
it is possible to show that:
\begin{equation}
(\rho^{Q}_{i} - \rho^{c}_{i}) \, t^{Q}_{i} + (\rho^{m} - \rho^{c}_{i}) \, t^{m}_{i} +
(\rho^{w} - \rho^{c}_{i}) \, t^{w}_{i} + (\rho^{1}_{i} - \rho^{c}_{i}) \, t^{1}_{i} +
\dots + (\rho^{Q-1}_{i} - \rho^{c}_{i}) \, t^{Q-1}_{i} + \rho^{c}_{i} \, S_{0} =
\sigma_{0} \: .
\label{eq:lithostatic-stress-density-contrasts}
\end{equation}
In order to describe the pressure exerted by all columns forming the interpretation
model on the surface $S_{0}$, Equation \ref{eq:lithostatic-stress-density-contrasts}
can be written, in matrix notation, as follows:
\begin{equation}
\mathbf{M}^{Q} \mathbf{t}^{Q} + \mathbf{M}^{m} \mathbf{t}^{m} + \mathbf{M}^{w}
\mathbf{t}^{w} + \mathbf{M}^{1} \mathbf{t}^{1} + \dots + \mathbf{M}^{Q-1}
\mathbf{t}^{Q-1} + \boldsymbol{\rho}^{c} S_{0} = \sigma_{0} \mathbf{1} \: ,
\label{eq:lithostatic-stress-matrix}
\end{equation}
where $\mathbf{1}$ is an $N \times 1$ vector with all elements equal to one, 
$\mathbf{t}^{\alpha}$ are $N \times 1$ vectors with $i$-th element defined by the
thickness $t^{\alpha}_{i}$ of a prism forming the $i$-th column, 
$\alpha = w, 1, \dots, Q-1, Q, m$, and $\mathbf{M}^{Q}$, $\mathbf{M}^{m}$, 
$\mathbf{M}^{w}$, $\mathbf{M}^{1}$, \dots, $\mathbf{M}^{Q-1}$ are $N \times N$ diagonal
matrices with elements $ii$ of main diagonal are given by density contrasts
$(\rho^{Q}_{i} - \rho^{c}_{i})$, $(\rho^{m} - \rho^{c}_{i})$, 
$(\rho^{w} - \rho^{c}_{i})$, $(\rho^{1}_{i} - \rho^{c}_{i})$ and $\dots$,
$(\rho^{Q-1}_{i} - \rho^{c}_{i})$, respectively, and $\boldsymbol{\rho}^{c}$ is an 
$N \times 1$ vector containing the densities of the prisms
representing the crust. By applying the first-order Tikhonov regularization
\citep{aster-etal2005} to the constant vector $\sigma_{0} \mathbf{1}$, we obtain the
following expression:
\begin{equation}
\mathbf{R} \left( \mathbf{C} \mathbf{p} + \mathbf{D} \mathbf{t} \right) = \mathbf{0} 
\: ,
\label{eq:tik1-lithostatic-stress}
\end{equation}
where $\mathbf{0}$ is a vector with null elements and the remaining terms are given by:
\begin{equation}
\mathbf{C} = \begin{bmatrix}
\mathbf{M}^{Q} & \mathbf{M}^{m} & \mathbf{0}
\end{bmatrix}_{N \times M} \: ,
\label{eq:matrix-C}
\end{equation}
\begin{equation}
\mathbf{D} = \begin{bmatrix}
\mathbf{M}^{w} & \mathbf{M}^{1} & \cdots & \mathbf{M}^{Q-1} & \boldsymbol{\rho}^{c}
\end{bmatrix}_{N \times \left( QN + 1 \right)} \: ,
\label{eq:matrix-D}
\end{equation}
\begin{equation}
\mathbf{t} = \begin{bmatrix}
\mathbf{t}^{w} \\ \mathbf{t}^{1} \\ \vdots \\ \mathbf{t}^{Q-1} \\ S_{0}
\end{bmatrix}_{\left( QN + 1 \right) \times 1}\: ,
\label{eq:vector-t}
\end{equation}
$\mathbf{p}$ is the parameter vector (Equation \ref{eq:parameter-vector}) and
$\mathbf{R}$ is an $\left( N-1 \right) \times N$ matrix, whose element $ij$ is defined
as follows:
\begin{equation}
\left[ \mathbf{R} \right]_{ij} = \begin{cases}
1 &, \quad j = i \\
-1 &, \quad j = i + 1 \\
0 &, \quad \text{otherwise}
\end{cases} \quad .
\label{eq:matrix-R}
\end{equation}
Finally, from Equation \ref{eq:tik1-lithostatic-stress}, it is possible to
define the regularizing function $\Psi_{0}(\mathbf{p})$ 
(Equation \ref{eq:goal-function}): 
\begin{equation}
\Psi_{0}(\mathbf{p}) = \| \mathbf{R} \left( \mathbf{C} \mathbf{p} + \mathbf{D}
\mathbf{t} \right) \|_{2}^{2} \: .
\label{eq:airy-constraint-function}
\end{equation}
We call this function as \textit{Airy constraint}. Notice that minimizing this
function imposes smoothness on the pressure exerted by the interpretation model on 
the isostatic compensation surface $S_{0}$.


\subsection{Smoothness constraint}


This constraint imposes smoothness on the adjacent thickness of the prisms forming the
deepest portion of the second layer and the shallowest part of the fourth layer
of the interpretation model by applying the first-order Tikhonov regularization
\citep{aster-etal2005} to the vectors $\mathbf{t}^{Q}$ and $\mathbf{t}^{m}$ 
(Equation \ref{eq:parameter-vector}). 
Mathematically, this constraint is represented by the regularizing function
$\Psi_{1}(\mathbf{p})$ (Equation \ref{eq:goal-function}):
\begin{equation}
\Psi_{1}(\mathbf{p}) = \| \mathbf{S}\mathbf{p} \|_{2}^{2} \: ,
\label{eq:smootheness-contraint}
\end{equation}
where $\mathbf{S}$ is an $\left( N-1 \right) \times M$ matrix given by:
\begin{equation}
\mathbf{S} = \begin{bmatrix}
\mathbf{R} & \mathbf{R} & \mathbf{0}
\end{bmatrix} \: ,
\label{eq:matrix-S}
\end{equation}
where $\mathbf{R}$ is defined by Equation \ref{eq:matrix-R} and $\mathbf{0}$
is a vector with all elements equal to zero.


\subsection{Equality constraint}

\subsubsection*{Equality constraint on basement depths}

Let $\mathbf{a}$ be a vector whose $k$-th element $a_{k}$,
$k = 1, \dots, A$, is the known basement depth at the horizontal coordinate
$y^{A}_{k}$ of the profile. These known basement depth values are used to define 
the regularizing function $\Psi_{2}(\mathbf{p})$ 
(Equation \ref{eq:goal-function}):
\begin{equation}
\Psi_{2}(\mathbf{p}) = \| \mathbf{A}\mathbf{p} - \mathbf{a} \|_{2}^{2} \: ,
\label{eq:equality-constraint-basement}
\end{equation}
where $\mathbf{A}$ is an $A \times M$ matrix whose $k$-th line has one element 
equal to one and all the remaining elements equal to zero. The location of the
single non-null element in the $k$-th line of $\mathbf{A}$ depends on the coordinate
$y^{A}_{k}$ of the known basement depth $a_{k}$. Let us consider, 
for example, an interpretation model formed by $N = 10$ columns. Consider also that 
the basement depth at the coordinates $y^{A}_{1} = y_{4}$ and $y^{A}_{2} = y_{9}$ of
the profile are equal to $25$ and $35.7$ km, respectively. In this case, $A = 2$,
$\mathbf{a}$ is a $2 \times 1$ vector with elements $a_{1} = 25$ and $a_{2} = 35.7$
and $\mathbf{A}$ is a $2 \times M$ matrix ($M = 2N + 1 = 21$). The element $4$ of the
first line and the element $9$ of the second line of $\mathbf{A}$ are equal to $1$ and
all its remaining elements are equal to zero.

\subsubsection*{Equality constraint on Moho depths}

Let $\mathbf{b}$ be a vector whose $k$-th element $b_{k}$,
$k = 1, \dots, B$, is the difference between the isostatic compensation depth
$S_{0}$ and the known Moho depth at the horizontal coordinate $y^{B}_{k}$ of the
profile. These differences, which must be positive, are used to define the 
regularizing function $\Psi_{3}(\mathbf{p})$ (Equation \ref{eq:goal-function}):
\begin{equation}
\Psi_{3}(\mathbf{p}) = \| \mathbf{B}\mathbf{p} - \mathbf{b} \|_{2}^{2} \: ,
\label{eq:equality-constraint-moho}
\end{equation}
where $\mathbf{B}$ is a $B \times M$ matrix whose $k$-th line has one element 
equal to one and all the remaining elements equal to zero. This matrix is defined 
in the same way as matrix $\mathbf{A}$ (Equation \ref{eq:equality-constraint-basement}).


\section{Conclusions}

\lipsum[1-3]


%%%%%%%%%%%%%%%%%% Figures %%%%%%%%%%%%%%%%%%%%%%%%%%%%%%%%%%%%%%%%%%%%%%%

%\plot{fig1}{width=0.6\textwidth}{
%%\plot{fig1}{width=\columnwidth}{
%	{caption \lipsum}
%	\label{fig:fig1}
%}

%%%%%%%%%%%%%%%%%% Figures %%%%%%%%%%%%%%%%%%%%%%%%%%%%%%%%%%%%%%%%%%%%%%%


\section{ACKNOWLEDGMENTS}

\lipsum[0-1]


\newpage

\bibliographystyle{seg.bst}  % style file is seg.bst
\bibliography{bib-file.bib}


\end{document}

