%\documentclass[paper,twocolumn,twoside]{geophysics}
%\documentclass[manuscript,revised]{geophysics}
\documentclass[manuscript]{geophysics}

% An example of defining macros
\newcommand{\rs}[1]{\mathstrut\mbox{\scriptsize\rm #1}}
\newcommand{\rr}[1]{\mbox{\rm #1}}

% Extra packages
\usepackage{amsmath}
\usepackage{bm}
\usepackage[pdftex,colorlinks=true]{hyperref}
\hypersetup{
	allcolors=black,
}
\usepackage{lipsum}

\renewcommand{\figdir}{figures} % figure directory

\begin{document}

\title{2D gravity inversion with isostatic constraint applied to passive rifted margins}

\renewcommand{\thefootnote}{\fnsymbol{footnote}} 

\ms{GEO-XXXX} % manuscript number

\address{
\footnotemark[1]Observat\'{o}rio Nacional, \\
Department of Geophysics, \\
Rio de Janeiro, Brazil}
\author{B. Marcela S. Bastos\footnotemark[1] and Vanderlei C. Oliveira Jr\footnotemark[1]}

\footer{vanderlei@on.br}
\lefthead{Bastos and Oliveira Jr.}
\righthead{2D gravity inversion for passive rifted margins}

\maketitle

\begin{abstract}

\lipsum[1]

\end{abstract}

\section{Introduction}

\lipsum[1-5]

\section{Methodology}


\subsection{Forward problem}

Let $\mathbf{d}^{o}$ be the observed data vector, whose $i$-th element $d^{o}_{i}$, 
$i = 1, \dots, N$, represent the observed gravity disturbance at the point 
$(x_{i}, y_{i}, z_{i})$, on a profile located over a rifted passive margin. The
coordinates are referred to a topocentric Cartesian system, with $z$ axis pointing
down, $y$-axis along the profile and $x$-axis perpendicular to the profile. Consider
that Figures \ref{fig:mass-distribution} and \ref{fig:reference-mass-distribution} 
show, respectively, the actual mass distribution and the reference mass 
distribution in the subsurface. In this case, we implicitly assume that the 
observed gravity disturbance is produced by an anomalous mass distribution defined 
as the difference between the mass distributions shown in Figures 
\ref{fig:mass-distribution} and \ref{fig:reference-mass-distribution}. We 
approximate this anomalous mass distribution by an interpretation model 
(Figure  \ref{fig:interpretation-model}) formed by $N$ columns of horizontally 
adjacent prisms having an infinite length along the $x$-axis, which is 
perpendicular to the profile. For convenience, we presume that 
the observed data a regularly spaced along the profile, so that there is one 
observation at the top of each column.

The $i$-th column of the interpretation model 
(Figure \ref{fig:interpretation-model}) is formed by four vertically adjacent 
layers. The first one is the shallowest, represents the water layer, is formed by 
a single prism, has thickness $t^{w}_{i}$ and a constant density contrast 
$\Delta \rho^{w} = \rho^{w} - \rho^{r}$, where $\rho^{w}$ and $\rho^{r}$ 
represents, respectively, the densities of water and the reference mass 
distribution (Figure \ref{fig:reference-mass-distribution}) at the same point. 

PAREI AQUI

A terceira camada representa a crosta, também é formada por apenas um prisma, possui espessura $t^{c}_{i}$ e contraste de densidade $\Delta \rho^{c}_{i} = \rho^{c} - \rho^{r}$, sendo $\rho^{c}$ a densidade da crosta. Por simplicidade, consideramos que a densidade nesta camada pode assumir dois valores, sendo um representativo da crosta continental, $\rho^{c}_{i} = \rho^{cc}$, $y_{i} <= y_{COT}$, e o outro da crosta oceânica, $\rho^{c}_{i} = \rho^{oc}$, $y_{i} > y_{COT}$, em que $y_{COT}$ representa a posição de uma superfície vertical que define a transição entre a crosta continental e oceânica ($COT$, do inglês “Continental-Ocean Transition”). Consequentemente, o contraste de densidade nesta camada pode assumir dois valores: $\Delta \rho^{c}_{i} = \rho^{cc} - \rho_{r}$ ou $\Delta \rho^{c}_{i} = \rho^{oc} - \rho_{r}$, a depender da posição em relação a $COT$. O topo desta terceira camada define a profundidade do embasamento e a base define a profundidade da Moho. A quarta camada representa o manto, é subdividida em duas partes, sendo cada uma formada por um prisma com a mesma densidade $\rho^{m}$ e, consequentemente, com o mesmo contraste de densidade $\Delta \rho^{m} = \rho^{m} - \rho {r}$. A porção mais rasa da quarta camada possui espessura $t^{m}_{i}$. Seu topo e base definem, respectivamente, as profundidades da Moho e da superfície de compensação isostática $S_{0}$. A parte mais profunda da quarta camada possui espessura $\Delta S_{0}$, topo na superfície de compensação isostática $S_{0}$ e a base na superfície $S_{0} + \Delta S_{0}$, que define a Moho no modelo de distribuição de densidades de referência (Figura 2.5). Por último, a segunda camada da $i$-ésima coluna do modelo interpretativo, $i = 1, …, N$, é definida pelo intérprete, de acordo com o ambiente geológico a ser caracterizado e a disponibilidade de informações a priori. De maneira geral, a segunda camada da $i$-ésima coluna do modelo interpretativo pode ser representada por um conjunto de $Q$ prismas verticalmente adjacentes, cada um com uma espessura $t^{q}_{i}$, densidade $\rho^{q}$ e um contraste de densidade  $\Delta \rho^{q} = \rho^{q} - \rho^{r}$, $q = 1, …, Q$.

Dessa forma, dados os valores de contraste de densidade, posição $y_{COT}$ da COT, a superfície de compensação isostática $S_{0}$, as espessuras dos prismas que formam a camada de água e as espessuras dos $Q – 1$ prismas que formam as porções mais rasas da segunda camada de prismas, é possível descrever o modelo interpretativo em termos de um vetor de parâmetros $M \times 1$, $M = 2N + 1$, dado por:

# Esta deve ser a Equação 3.1
\begin{equation}
\mathbf{p} = \begin{bmatrix}
\mathbf{t}^{Q} \\
\mathbf{t}^{m} \\
\Delta S_{0}
\end{bmatrix} \: ,
\end{equation}

em que $\mathbf{t}^{Q}$ é um vetor $N \times 1$, cujo $i$-ésimo elemento $t^{Q}_{i}$ representa a espessura do prisma que forma a porção mais profunda da segunda camada da $i$-ésima coluna do modelo interpretativo, e $\mathbf{t}^{m}$ é um vetor $N \times 1$, cujo $i$-ésimo elemento $t^{m}_{i}$ representa a espessura do prisma que forma a porção mais rasa da quarta camada da $i$-ésima coluna do modelo interpretativo.

Nesse caso, o distúrbio de gravidade predito pelo modelo interpretativo na posição $(x_{i}, y_{i}, z_{i})$ pode ser escrito a partir da Equação 2.2 da seguinte forma:

# Esta deve ser a Equação 3.2
\begin{equation}
d_{i}(\mathbf{p}) = k_{g} \, G \, \sum_{j = 1}^{L} f_{ij}(\mathbf{p}) \: ,
\end{equation}

em que $f_{ij}(\mathbf{p})$ representa a integral tripla da Equação 2.2 avaliada no volume do $j$-ésimo prisma do modelo interpretativo e $L$ representa o total de prismas que formam todas as camadas do modelo interpretativo. Neste trabalho, as integrais $f_{ij}(\mathbf{p})$ foram calculadas pelas expressões propostas por NAGY et al. (2000), usando o pacote Fatiando a Terra (UIEDA et al., 2013).


\subsection{Inverse problem}


Seja $\mathbf{d}(\mathbf{p})$ o vetor de dados preditos, cujo $i$-ésimo elemento $d_{i}(\mathbf{p})$ representa o dado predito pelo modelo interpretativo na posição $(x_{i}, y_{i}, z_{i})$ (Equação 3.2) em função do vetor de parâmetros $\mathbf{p}$ (Equação 3.1).

O problema de estimar o vetor de parâmetros $\hat{\mathbf{p}}$ que produz os dados preditos $\mathbf{d}(\mathbf{p})$ mais próximos aos dados observados $\mathbf{d}^{o}$ pode ser formulado como um problema inverso vinculado, não-linear, que consiste em minimizar a função

# Esta deve ser a Equação 4.1
\begin{equation}
\Gamma (\mathbf{p}) = \Phi(\mathbf{p}) + \mu \sum_{k = 0}^{3} \alpha_{k} \Psi(\mathbf{p}) \: ,
\end{equation}

sujeita a condição de que todos os elementos do vetor de parâmetros sejam positivos. Na Equação 4.1, $\mu$ representa o parâmetro de regularização, $\Phi(\mathbf{p})$ representa a função do ajuste dada por

# Esta deve ser a Equação 4.2
\begin{equation}
\Phi(\mathbf{p}) = \frac{1}{N} \| \mathbf{d}^{o} - \mathbf{d}(\mathbf{p}) \|_{2}^{2} \: , 
\end{equation}

em que $\| \cdot \|_{2}^{2}$ representa o quadrado da norma Euclidiana, $\alpha_{k}$ representam os pesos atribuídos às funções $\Psi(\mathbf{p})$, que definem os vínculos, $k = 0, 1, 2, 3$.

\subsection{Airy constraint}

Considere que o modelo de distribuição de massas anômalas (Figura 2.2) e, consequentemente, o modelo interpretativo (Figura 3.1) estejam em equilíbrio isostático, de acordo com o modelo de Airy. Neste caso, a pressão exercida por estes modelos deve ser constante sobre a superfície de compensação isostática $S_{0}$. Esta condição sobre a pressão também é válida para qualquer interface paralela a $S_{0}$ e que esteja localizada no manto em uma profundidade maior.

A pressão exercida pela $i$-ésima coluna do modelo interpretativo em um ponto sobre a superfície $S_{0}$ é dada por:

# Esta deve ser a Equação 4.3
\begin{equation}
t^{w}_{i} \rho^{w} + t^{1}_{i} \rho^{1}_{i} + \dots + t^{Q}_{i} \rho^{Q}_{i} + t^{c}_{i} \rho^{c}_{i} + t^{m}_{i} \rho^{m} = \sigma_{0} \: ,
\end{equation}

em que $\sigma_{0}$ é uma constante positiva e arbitrária,  $\rho^{w}$ é a densidade da camada de água, $\rho^{q}_{i}$, $q = 1, \dots, Q$, são as densidades das porções que formam a segunda camada do modelo interpretativo, $\rho^{c}_{i}$ é a densidade da terceira camada, que pode assumir valor de crosta continental $\rho^{cc}$ ou oceânica $\rho^{oc}$ a depender da posição em relação a COT, e  $\rho^{m}$ é a densidade do manto. Rearranjando os termos da Equação 4.3 e utilizando a relação 

# Esta deve ser a Equação 4.4
\begin{equation}
S_{0} = t^{w}_{i} + t^{1}_{i} + \dots + t^{Q}_{i} + t^{c}_{i} + t^{m}_{i} \: ,
\end{equation}

é possível mostrar que:

# Esta deve ser a Equação 4.5
\begin{equation}
(\rho^{Q}_{i} - \rho^{c}_{i}) \, t^{Q}_{i} + (\rho^{m} - \rho^{c}_{i}) \, t^{m}_{i} + (\rho^{w} - \rho^{c}_{i}) \, t^{w}_{i} + (\rho^{1}_{i} - \rho^{c}_{i}) \, t^{1}_{i} + \dots + (\rho^{Q-1}_{i} - \rho^{c}_{i}) \, t^{Q-1}_{i} + \rho^{c}_{i} \, S_{0} = \sigma_{0} \: .
\end{equation}

Com o intuito de descrever a pressão exercida por todas as $N$ colunas do modelo interpretativo sobre a superfície $S_{0}$, a Equação 4.5 pode ser reescrita, em notação matricial, da seguinte forma:

# Equação 4.6
\begin{equation}
\mathbf{M}^{Q} \mathbf{t}^{Q} + \mathbf{M}^{m} \mathbf{t}^{m} + \mathbf{M}^{w} \mathbf{t}^{w} + \mathbf{M}^{1} \mathbf{t}^{1} + \dots + \mathbf{M}^{Q-1} \mathbf{t}^{Q-1} + \boldsymbol{\rho}^{c} S_{0} = \sigma_{0} \mathbf{1} \: ,
\end{equation}

em que $\mathbf{1}$ é um vetor $N \times 1$ com todos os elementos iguais a 1, $\mathbf{t}^{\alpha}$ são vetores $N \times 1$ com $i$-ésimo elemento definido pela espessura $t^{\alpha}_{i}$ da coluna $i$, $\alpha = w, 1, \dots, Q-1, Q, m$, e $\mathbf{M}^{Q}$, $\mathbf{M}^{m}$, $\mathbf{M}^{w}$, $\mathbf{M}^{1}$, \dots, $\mathbf{M}^{Q-1}$ são matrizes diagonais $N \times N$ com o $i$-ésimo elemento da diagonal dado por, respectivamente, $(\rho^{Q}_{i} - \rho^{c}_{i})$, $(\rho^{m} - \rho^{c}_{i})$, $(\rho^{w} - \rho^{c}_{i})$, $(\rho^{1}_{i} - \rho^{c}_{i})$, $\dots$, $(\rho^{Q-1}_{i} - \rho^{c}_{i})$ e $\boldsymbol{\rho}^{c}$ é um vetor $N \times 1$ que contém a densidade dos prismas que representam a crosta. Aplicando o regularizador de Tikhonov de primeira ordem (ASTER et al., 2013) no vetor de pressões $\sigma_{0} \mathbf{1}$ exercidas pelo modelo interpretativo sobre a superfície $S_{0}$, obtemos a seguinte equação:

# Equação 4.7
\begin{equation}
\mathbf{R} \left( \mathbf{C} \mathbf{p} + \mathbf{D} \mathbf{t} \right) = \mathbf{0} \: ,
\end{equation}

em que $\mathbf{0}$ é um vetor $N \times 1$ com todos os elementos iguais a zero e os demais termos são dados por:

# Equação 4.8
\begin{equation}
\mathbf{C} = \begin{bmatrix}
\mathbf{M}^{Q} & \mathbf{M}^{m} & \mathbf{0}
\end{bmatrix}_{N \times \left( 2N + 1 \right)} \: ,
\end{equation}

# Equação 4.9
\begin{equation}
\mathbf{D} = \begin{bmatrix}
\mathbf{M}^{w} & \mathbf{M}^{1} & \cdots & \mathbf{M}^{Q-1} & \boldsymbol{\rho}^{c}
\end{bmatrix}_{N \times \left( QN + 1 \right)} \: ,
\end{equation


# Equação 4.10
\begin{equation}
\mathbf{t} = \begin{bmatrix}
\mathbf{t}^{w} \\ \mathbf{t}^{1} \\ \vdots \\ \mathbf{t}^{Q-1} \\ S_{0}
\end{bmatrix}_{\left( QN + 1 \right) \times 1}\: ,
\end{equation}

$\mathbf{p}$ é o vetor de parâmetros (Equação 3.1) e $\mathbf{R}$ é uma matriz $\left( N-1 \right) \times N$, cujo elemento $ij$ é definido da seguinte forma:

#Equation 4.11
\begin{equation}
\left[ \mathbf{R} \right]_{ij} = \begin{cases}
1 &, \quad j = i \\
-1 &, \quad j = i + 1 \\
0 &, \quad \text{otherwise}
\end{cases} \quad .
\end{equation}

Por fim, a partir da Equação 4.7, é possível definir o “Vínculo de Airy” da seguinte forma:

# Equation 4.12
\begin{equation}
\Psi_{0}(\mathbf{p}) = \| \mathbf{R} \left( \mathbf{C} \mathbf{p} + \mathbf{D} \mathbf{t} \right) \|_{2}^{2} \: .
\end{equation}

(sem pular linha) Note que este vínculo impõe a informação a priori de que a pressão exercida pelo modelo interpretativo sobre a superfície $S_{0}$ deve variar de forma suave.

\subsection{Smoothness constraint}

Este vínculo é conhecido como vínculo de Tikhonov de primeira ordem (ASTER et al., 2013) e impõe uma variação suave entre parâmetros adjacentes nos vetores $\mathbf{t}^{Q}$ e $\mathbf{t}^{m}$ (Equação 3.1). Ou seja, estabelece uma igualdade aproximada entre os pares de espessuras adjacentes na parte mais profunda da segunda camada e na parte mais rasa da quarta camada do modelo interpretativo (Figura 3.1). Matematicamente, este vínculo é representado pela seguinte expressão:

#Equation 4.13
\begin{equation}
\Psi_{1}(\mathbf{p}) = \| \mathbf{S}\mathbf{p} \|_{2}^{2} \: ,
\end{equation}

em que $\mathbf{S}$ é uma matriz $\left( 2N-2 \right) \times M$, cujo elemento $ij$ é definido da seguinte forma:

#Equation 4.14
\begin{equation}
\left[ \mathbf{S} \right]_{ij} = \begin{cases}
1 &, \quad j = i \\
-1 &, \quad j = i + 1 \\
0 &, \quad \text{otherwise}
\end{cases} \quad .
\end{equation}


\section{Conclusions}

\lipsum[1-3]


%%%%%%%%%%%%%%%%%% Figures %%%%%%%%%%%%%%%%%%%%%%%%%%%%%%%%%%%%%%%%%%%%%%%

%\plot{fig1}{width=0.6\textwidth}{
%%\plot{fig1}{width=\columnwidth}{
%	{caption \lipsum}
%	\label{fig:fig1}
%}

%%%%%%%%%%%%%%%%%% Figures %%%%%%%%%%%%%%%%%%%%%%%%%%%%%%%%%%%%%%%%%%%%%%%


\section{ACKNOWLEDGMENTS}

\lipsum[0-1]


%\newpage

\bibliographystyle{seg}  % style file is seg.bst

\bibliography{bib-file.bib} 

\end{document}

