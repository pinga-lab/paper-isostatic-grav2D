%\documentclass[paper,twocolumn,twoside]{geophysics}
%\documentclass[manuscript,revised]{geophysics}
\documentclass[manuscript]{geophysics}

% An example of defining macros
\newcommand{\rs}[1]{\mathstrut\mbox{\scriptsize\rm #1}}
\newcommand{\rr}[1]{\mbox{\rm #1}}

% Extra packages
\usepackage{amsmath}
\usepackage{bm}
\usepackage[pdftex,colorlinks=true]{hyperref}
\hypersetup{
	allcolors=black,
}
\usepackage{lipsum}
\usepackage[table]{xcolor}

\renewcommand{\figdir}{figures} % figure directory

\begin{document}

\title{Isostatic constraint for 2D gravity inversion on passive rifted margins}

\renewcommand{\thefootnote}{\fnsymbol{footnote}} 

\ms{GEO-XXXX} % manuscript number

%\address{
%\footnotemark[1]Observat\'{o}rio Nacional, \\
%Department of Geophysics, \\
%Rio de Janeiro, Brazil}
%\author{B. Marcela S. Bastos\footnotemark[1] and Vanderlei C. Oliveira Jr\footnotemark[1]}

%\footer{vanderlei@on.br}
%\lefthead{Bastos and Oliveira Jr.}
\righthead{Isostatic constraint for gravity inversion on passive margins}

\maketitle

%\begin{abstract}
%
%We propose a new gravity inversion method for jointly estimating the basement and 
%Moho reliefs along a profile crossing a passive rifted margin.
%We approximate the subsurface by an interpretation model composed of 
%adjacent columns, each one formed by vertically stacked prisms having constant 
%and known density contrasts.
%Our method consists in solving a non-linear inverse problem to estimate the 
%thickness of specific prisms defining the basement and Moho geometries, 
%as well as a constant thickness defining the maximum depth of our interpretation 
%model.
%This maximum depth represents a planar reference Moho.
%To obtain stable solutions, we impose smoothness on the basement and Moho relief,
%force them to be close to known depths along the profile and also impose
%isostatic equilibrium according to the Airy-Heiskanen model.
%Our method imposes isostatic equilibrium by constraining the lithostatic stress 
%exerted by the interpretation model at a given constant depth, 
%below which there are no lateral density variations.
%This isostatic constraint introduces the information that the lithostatic stress 
%is mostly smooth, except at some isolated regions, where it can present abrupt 
%variations. At these regions, our method enables the interpretation model to deviate
%from the isostatic equilibrium.
%Tests with synthetic data show the good performance of our method in
%determining the basement and Moho geometries at regions with  
%pronounced crustal thinning, which is typical of passive volcanic margins.
%Results obtained by our method at the Pelotas basin, considered a classical example
%of passive volcanic margin at the southern of Brazil, agree with a previous
%interpretation obtained independently by using ultra-deep seismic data.
%The applications to synthetic and real data show that our method is a
%promising tool for interpreting gravity data on passive rifted margins.
%
%
%\end{abstract}


\section{Applications to synthetic data - Sensitivity to the Moho reference thickness}

We tested the sensitivity to the Moho reference thickness by keeping constant the isostatic compensation depth ($41$ km) and show the results of steps 2 and 3. Parameters defining the interpretation model are shown in Table~\ref{tab:volcanic-margin-model}.  We used the same parameters for inversion used in the synthetic applications of the paper.

Figure~\ref{fig:volc-margem-step2} and Figure~\ref{fig:volc-margem-step3} shows the estimated model obtained at the Step 2 and Step 3 of our algorithm with initial approximation Moho reference thickness adopted in the applications of the paper ($49.5$ km). 

Figure~\ref{fig:volc-margem-step2-test1} (initial guess 1), Figure~\ref{fig:volc-margem-step2-test2} (initial guess 2) and Figure~\ref{fig:volc-margem-step2-test3} (initial guess 3) show the estimated model obtained at the end of Step 2 where the initial approximation Moho reference thickness was defined at $41.2$ km, $45$ km and $50.5$ km, respectively.  These models presented basement and Moho reliefs closer to the estimated surfaces in the Figure~\ref{fig:volc-margem-step2}. In the same way, these results (Figure~\ref{fig:volc-margem-step2-test1}, Figure~\ref{fig:volc-margem-step2-test2} and Figure~\ref{fig:volc-margem-step2-test3}) show predicted gravity disturbance and lithostatic stress curves with the same behavior that presented in the Figure~\ref{fig:volc-margem-step2}. 

Figure~\ref{fig:volc-margem-step3-test1} (initial guess 1), Figure~\ref{fig:volc-margem-step3-test2} (initial guess 2) and Figure~\ref{fig:volc-margem-step3-test3} (initial guess 3) show the estimated model obtained at the end of Step 3. We used different $\sigma$ constants for the tests. These models presented basement and Moho reliefs closer to the estimated surfaces in the Figure~\ref{fig:volc-margem-step3}. These results (Figures~\ref{fig:volc-margem-step3-test1}, Figure~\ref{fig:volc-margem-step3-test2} and ~\ref{fig:volc-margem-step3-test3}) show predicted gravity disturbance and lithostatic stress curves with the same behavior that presented in the Figure~\ref{fig:volc-margem-step3}. 





%%%%%%%%%%%%%%%%%% Tables %%%%%%%%%%%%%%%%%%%%%%%%%%%%%%%%%%%%%%%%%%%%%%%%

\tabl{volcanic-margin-model}{Properties of the volcanic margin model. 
	The model extends from $y = 0$ km to $y = 383$ km, the Continent-Ocean 
	Transition (COT) is located at $y_{COT} = 350$ km and the reference 
	Moho is located at $S_{0} + \Delta S = 43.2$ km, where 
	$\Delta S = 2.2$ km.
	The density contrasts $\Delta\rho^{(\alpha)}$ are defined with respect to the
	reference value $\rho^{(r)} = 2870$ kg/m$^{3}$, which coincides with
	the density $\rho^{(cc)}$ attributed to the continental crust.
	\label{tab:volcanic-margin-model}
}{
\begin{center}
	\begin{tabular}[]{lccc}
		\hline
		\textbf{Geological meaning} & $\rho^{(\alpha)}$ (kg/m$^{3}$) & $\Delta\rho^{(\alpha)}$ (kg/m$^{3}$) & $\alpha$ \\
		\hline
		water & $1030$ & $-1840$ & $w$ \\
		\hline
		sediments & $2350$ & $-520$ & $1$ \\
		SDR & $2855$ & $-15$ & $2$ \\
		\hline 
		continental crust & $2870$ & $0$ & $cc$ \\
		oceanic crust & $2885$ & $15$ & $oc$ \\
		\hline
		mantle & $3240$ & $370$ & $m$ \\
		\hline
	\end{tabular}
\end{center}
}


%%%%%%%%%%%%%%%%%% Tables %%%%%%%%%%%%%%%%%%%%%%%%%%%%%%%%%%%%%%%%%%%%%%%%

%%%%%%%%%%%%%%%%%% Figures %%%%%%%%%%%%%%%%%%%%%%%%%%%%%%%%%%%%%%%%%%%%%%%

%% Results

\plot{volcanic-margin-grafics-estimated-model-alphas_2_1_1_2}{width=0.6\textwidth}{
	{Application to synthetic data. Results obtained in Step 2.
	(Bottom panel) Estimated and true surfaces,
	initial basement and Moho used in the inversion (initial guess) and
	known depths at basement and Moho.
	(Middle panel) True and estimated lithostatic stress curves computed
	by using equation~\ref{eq:lithostatic-stress-densities}. The values are multiplied 
	by a constant gravity value equal to $9.81$ m/s$^{2}$.
	(Upper panel) Gravity disturbance data produced by the volcanic 
	margin model (simulated data), by the estimated model 
	(predicted data) and by the model used as initial guess in the 
	inversion (initial guess data).
	The contour of the prisms forming the interpretation model were omitted.
	The density contrasts were defined according to Table~\ref{tab:volcanic-margin-model}.}
	\label{fig:volc-margem-step2}
}

\plot{volcanic-margin-grafics-estimated-model-alphas_2_1_1_2-sgm_22}{width=0.6\textwidth}{
	{Application to synthetic data. Results obtained in Step 3 by using 
	$\sigma = 22$ (equation~\ref{eq:elements-wii}).
	The remaining informations are the same shown in the caption of
	Figure~\ref{fig:volc-margem-step2}.
	\label{fig:volc-margem-step3}}
}

\plot{volcanic-margin-grafics-estimated-model-alphas_2_1_1_2-T1-dS0_200}{width=0.6\textwidth}{
	{Application to synthetic data (initial guess 1: $S0 + \Delta S0 = 41.2$ km). Results obtained in Step 2.
	The remaining informations are the same shown in the caption of
	Figure~\ref{fig:volc-margem-step2}.
	\label{fig:volc-margem-step2-test1}}
}

\plot{volcanic-margin-grafics-estimated-model-alphas_2_1_1_2-T1-dS0_200-sgm_17}{width=0.6\textwidth}{
	{Application to synthetic data (initial guess 1: $S0 + \Delta S0 = 41.2$ km). Results obtained in Step 3 by using 
	$\sigma = 17$ (equation~\ref{eq:elements-wii}).
	The remaining informations are the same shown in the caption of
	Figure~\ref{fig:volc-margem-step2}.
	\label{fig:volc-margem-step3-test1}}
}

\plot{volcanic-margin-grafics-estimated-model-alphas_2_1_1_2-T1-dS0_4000}{width=0.6\textwidth}{
	{Application to synthetic data (initial guess 2: $S0 + \Delta S0 = 45$ km). Results obtained in Step 2.
	The remaining informations are the same shown in the caption of
	Figure~\ref{fig:volc-margem-step2}.
	\label{fig:volc-margem-step2-test2}}
}

\plot{volcanic-margin-grafics-estimated-model-alphas_2_1_1_2-T1-dS0_4000-sgm_22}{width=0.6\textwidth}{
	{Application to synthetic data (initial guess 2: $S0 + \Delta S0 = 45$ km). Results obtained in Step 3 by using 
	$\sigma = 22$ (equation~\ref{eq:elements-wii}).
	The remaining informations are the same shown in the caption of
	Figure~\ref{fig:volc-margem-step2}.
	\label{fig:volc-margem-step3-test2}}
}

\plot{volcanic-margin-grafics-estimated-model-alphas_2_1_1_2-T1-dS0_9500}{width=0.6\textwidth}{
	{Application to synthetic data (initial guess 3: $S0 + \Delta S0 = 50.5$ km). Results obtained in Step 2.
	The remaining informations are the same shown in the caption of
	Figure~\ref{fig:volc-margem-step2}.
	\label{fig:volc-margem-step2-test3}}
}

\plot{volcanic-margin-grafics-estimated-model-alphas_2_1_1_2-T1-dS0_9500-sgm_22}{width=0.6\textwidth}{
	{Application to synthetic data (initial guess 3: $S0 + \Delta S0 = 50.5$ km). Results obtained in Step 3 by using 
	$\sigma = 22$ (equation~\ref{eq:elements-wii}).
	The remaining informations are the same shown in the caption of
	Figure~\ref{fig:volc-margem-step2}.
	\label{fig:volc-margem-step3-test3}}
}


%%%%%%%%%%%%%%%%%% Figures %%%%%%%%%%%%%%%%%%%%%%%%%%%%%%%%%%%%%%%%%%%%%%%

%\section{ACKNOWLEDGMENTS}

%\lipsum[1]


\newpage

\bibliographystyle{seg.bst}  % style file is seg.bst
\bibliography{bib-file.bib}


\end{document}

