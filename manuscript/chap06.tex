\chapter{Conclusões}

Apresentamos uma metodologia que inverte, simultaneamente, em três etapas os relevos do embasamento e da descontinuidade de Mohorovicic para uma bacia de margem passiva, utilizando dados de distúrbio de gravidade. Os vínculos aplicados no processo são efetivos em manter a estabilidade da inversão, e além disso, os testes sintéticos mostram que o vínculo isostático pode consideravelmente melhorar o modelo estimado em regiões que apresentam pronunciado afinamento crustal, característica típica de bacias de margem passiva vulcânicas. 

O método estima os relevos do embasamento e Moho sob a premissa que os contrastes de densidades das camadas do modelo $\Delta \rho^{(\alpha)}$, $\alpha = w, 1, \dots, Q, cc, oc, m$ e todas as superfícies acima do embasamento são conhecidos. Fixamos também uma profundidade $S_{0}$ (superfície de compensação isostática) a partir da qual não existem variações laterais de densidade e onde a pressão litostática é constante. A posição ao longo do perfil ($y_{COT}$) onde ocorre a transição entre as crostas continental e oceânica ($COT$) também é conhecida. O modelo interpretativo é aproximado como um conjunto de colunas de prismas, onde cada coluna representa as quatro camadas do modelo divididas em prismas verticalmente adjacentes. A primeira camada (água) é representada por um único prisma. A segunda camada é representada por $Q$ partes (sedimentos, sal e/ou vulcânicas), sendo essa quantidade definida pelo intérprete de acordo com a área de estudo. A terceira camada é formada por um prisma que, de acordo com sua posição ao longo do perfil, pode representar crosta continental ou oceânica. A quarta e última camada define o manto e é subdividida em duas partes, estando a mais rasa acima da superfície $S_{0}$ e a mais profunda abaixo desta mesma superfície. Cada prisma que forma o modelo interpretativo possui contraste de densidade constante e é definido com relação a um modelo de distribuição de massa de referência. Este modelo possui duas camadas separadas por uma superfície planar chamada Moho de referência. A camada acima desta superfície representa uma crosta homogênea com densidade de referência igual a densidade da crosta continental e a camada mais profunda representa um manto homogêneo. A superfície Moho de referência forma a base do modelo interpretativo. O vetor de parâmetros estimado na inversão é formado pelas espessuras dos prismas que formam a porção mais profunda da segunda camada ($t^{(Q)}_{i}$), cuja base é o relevo do embasamento, pelas espessuras dos prismas que formam a porção mais rasa da quarta camada ($t^{(m)}_{i}$), cujo topo é o relevo da Moho e a base é a superfície $S_{0}$ e pela espessura da porção mais profunda da quarta camada ($\Delta S_{0}$), cujo topo é a superfície $S_{0}$ e a base é a superfície Moho de referência ($S_{0} + \Delta S_{0}$). Para garantir a estabilidade e diminuir a ambiguidade da solução, introduzimos quatro vínculos na função objetivo: vínculo isostático, vínculo de suavidade (Tikhonov de primeira ordem), vínculo de igualdade sobre o vetor de espessuras $\mathbf{t}^{(Q)}$ e vínculo de igualdade sobre o vetor de espessuras $\mathbf{t}^{(m)}$. No Passo 1 do método são definidos os parâmetros que descrevem o modelo interpretativo ($\Delta \rho^{(\alpha)}$, $\alpha = w, 1, \dots, Q, cc, oc, m$, $y_{COT}$ e $S_{0}$) e os parâmetros que controlam a inversão, como: pesos associados aos vínculos de suavidade e igualdade ($\alpha_{\ell}$, $\ell = 1, 2, 3$), limites inferior e superior ($p^{min}_{j}$ e $p^{max}_{j}$) para o vetor de parâmetros a ser estimado, vetores contendo as espessuras conhecidas ($\mathbf{a}$ e $\mathbf{b}$) e uma aproximação inicial para o vetor de parâmetros ($\mathbf{p}^{(0)}$), que deve satisfazer o vínculo de desigualdade. No Passo 2 é utilizada a mesma aproximação inicial que foi utilizada no Passo 1 ($\mathbf{p}^{(0)}$) e, além dos vínculos de suavidade ($\Psi_{1}(\mathbf{p})$) e igualdade ($\Psi_{2}(\mathbf{p})$ e $\Psi_{3}(\mathbf{p})$), é também aplicado o vínculo isostático ($\Psi_{0}(\mathbf{p})$) de forma plena ao longo de todo o perfil para estimar um novo vetor de parâmetros preditos ($\mathbf{p}^{(2)}$). Já no Passo 3, o vetor de parâmetros estimado no Passo 2 ($\mathbf{p}^{(2)}$) é utilizado como aproximação inicial para inversão, e o vínculo isostático é aplicado em diferentes quantidades ao longo do perfil dependendo da diferença entre os dados simulado e o predito no Passo 2. O Passo 3 do nosso método é útil em permitir que o modelo interpretativo se desvie do equilíbrio isostático em regiões em que o uso do vínculo isostático, no Passo 2, produz grandes diferenças entre os dados de distúrbio simulado e predito.

Realizamos testes em quatro diferentes modelos sintéticos de bacias de margem passiva, sendo que dois destes modelos simulam bacias do tipo vulcânica e do tipo pobres em magmas. Os dois modelos que simulam bacias simples não possuem subdivisão em sua segunda camada, sendo a mesma representativa de sedimentos siliciclásticos. Já o modelo que simula uma bacia de margem passiva vulcânica, tem a porção superior de sua segunda camada representada por sedimentos siliciclásticos e a porção inferior por vulcânicas, enquanto que o modelo que simula uma bacia de margem passiva pobre em magmas tem a porção superior subdividida em sedimentos siliciclásticos e sal e a porção inferior representada por sedimentos do tipo pré-sal. Para os dois modelos simples não foi necessário aplicar o Passo 3 do nosso método, visto que os modelos estimados produziram dados preditos com muito bom ajuste com relação aos dados verdadeiros. No entanto, estes modelos estimados com diferentes geometrias para os relevos do embasamento e da Moho foram obtidos com diferentes combinações no uso dos vínculos isostáticos, ficando claro a ambiguidade na solução do problema inverso. Com os testes realizados sobre o modelo interpretativo do tipo margem vulcânica, no Passo 2 do nosso método, verificamos a importância do uso do vínculo isostático em recuperar o relevo do embasamento em regiões com pronunciado afinamento crustal, produzindo, no entanto, grandes diferenças entre os dados predito e verdadeiro. Estes resíduos entre os dados de distúrbio de gravidade estão de acordo com o fato do equilíbrio isostático não poder ser perfeitamente explicado pelo modelo de Airy ao longo de uma margem continental. Já o Passo 3 do nosso método foi efetivo em neutralizar este problema, permitindo que o modelo interpretativo se desviasse do equilíbrio isostático nestas regiões de altos resíduos entre os dados de distúrbio de gravidade, e então recuperando melhor os relevos do embasamento e Moho com limitada quantidade de informação \textit{a priori}. 

Nossa metodologia pode ser usada de modo a automatizar o processo já em uso com modelagem direta para mapeamento regional dos relevos do embasamento e da Moho, podendo ainda, ser aperfeiçoada com a inclusão de variação lateral de densidade e a expanção do método para modelos interpretativos 3D.

