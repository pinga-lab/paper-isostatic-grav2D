%\documentclass[paper,twocolumn,twoside]{geophysics}
%\documentclass[manuscript,revised]{geophysics}
\documentclass[manuscript]{geophysics}

% An example of defining macros
\newcommand{\rs}[1]{\mathstrut\mbox{\scriptsize\rm #1}}
\newcommand{\rr}[1]{\mbox{\rm #1}}

% Extra packages
\usepackage{amsmath}
\usepackage{bm}
\usepackage[pdftex,colorlinks=true]{hyperref}
\hypersetup{
	allcolors=black,
}
\usepackage{lipsum}
\usepackage[table]{xcolor}

\renewcommand{\figdir}{figures} % figure directory

\begin{document}

\title{Isostatic constraint for 2D gravity inversion on passive rifted margins}

\renewcommand{\thefootnote}{\fnsymbol{footnote}} 

\ms{GEO-XXXX} % manuscript number

\address{
\footnotemark[1]Observat\'{o}rio Nacional, \\
Department of Geophysics, \\
Rio de Janeiro, Brazil}
\author{B. Marcela S. Bastos\footnotemark[1] and Vanderlei C. Oliveira Jr\footnotemark[1]}

\footer{vanderlei@on.br}
\lefthead{Bastos and Oliveira Jr.}
\righthead{Isostatic constraint for gravity inversion on passive margins}

\maketitle

\begin{abstract}

\lipsum[1]

\end{abstract}

\section{Introduction}

Several methods have been proposed for using gravity and/or magnetic data to estimate
the boundaries of juxtaposed sedimentary layers, the relief of basement under sedimentary 
basins and/or the Mohorovicic discontinuity (or simply Moho), which separates crust and 
mantle.
These geophysical discontinuities represent, for such particular methods, 
density and/or magnetization contrasts in subsurface.
All these methods suffer from the inherent ambiguity \citep{roy1962, skeels1962} in
determining the true physical property distribution that produces a discrete set of
observed potential-field data. 
It is well known that, by using different density and/or magnetization contrasts, 
it is possible to
find different surfaces producing the same potential-field data. 
To partially overcome this problem and obtain meaningful solutions, the interpreter
must commonly use priori information obtained from seismic data and/or boreholes in
order to constrain the range of possible models.

There are methods that approximate the subsurface by a grid of
juxtaposed cells with constant physical property. 
They estimate the physical property value of each cell and then 
interpret the estimated values to indirectly estimate the geometry of the 
geophysical discontinuities.
Although very useful in geophysics, such methods are
outside the scope of the present work.
Here, we consider methods that represent discontinuities by surfaces 
separating layers with constant or depth-dependent physical property distribution 
(density and/or magnetization).
In this case, the geometry of the geophysical discontinuities are directly determined
by estimating the geometrical parameters describing the surfaces.

Different criteria can be used to classify the methods that directly estimate
the geometry of geophysical discontinuities.
Those applied over a sedimentary basin, for example, can be considered local scale methods,
whereas those applied over a continent or country can be considered regional scale methods
and those applied over the whole globe can be considered global scale methods.
They can also be classified according to the number of geophysical surfaces
to be estimated.

By using these criteria, it is possible to define a first group of methods
estimating the geometry of a single interface.
In this group, there are local scale methods in space domain
% scale: local | data: gravity | domain: space
\citep[e.g.,][]{bott1960, tanner1967, cordell-henderson1968, dyrelius-vogel1972, pedersen1977,
pilkington-crossley1986, richardson-macinnes1989, barbosa-etal1997, 
barbosa-etal1999, barbosa-etal1999b, silva-etal2006, pilkington2006, 
chakravarthi-sundararajan2007, martins-etal2010, silva-etal2010, lima-etal2011, 
martins-etal2011, barnes-barraud2012, silva-etal2014, silva-santos2017},
and Fourier domain
% scale: local | data: gravity | domain: Fourier
\citep[e.g.,][]{oldenburg1974, granser1987, reamer-ferguson1989, guspi1993}.
Most of these methods were applied to estimate the relief of basement under
a sedimentary basin.
There are also regional scale methods for estimating a single interface 
representing the Moho in spaced domain 
% scale: regional | data: gravity | domain: space
\citep[e.g.,][]{shin-etal2009, bagherbandi-eshagh2012, barzaghi-biagi2014, sampietro2015, uieda-barbosa2017} and in Fourier domain 
% scale: regional | data: gravity | domain: Fourier
\citep[e.g.,][]{braitenberg-etal1997, braitenberg-zadro1999, vandermeijde-etal2013}.
Additionally, there are some global scale methods for estimating the Moho
in spaced domain
% scale: global | data: gravity | domain: space
\citep[e.g.,][]{sunkel1985, sjoberg2009}.

The second group of methods is formed by those estimating multiple surfaces
separating layers with constant or depth-dependent physical properties 
\citep[e.g.,][]{pilkington-crossley1986b, gallardo-etal2005, camacho-etal2011, 
salem-etal2014}.
All these methods have been applied at local scale, to characterize a single 
sedimentary basin.
The number of methods forming this group is significantly lower than that
in the first one.
Additionally, the methods forming the second group suffer from a greater 
ambiguity and, as a consequence, they require more priori information to
decrease the number of possible solutions.

Among those directly estimating the geometry of the surfaces, 
there are some regional and global scale methods in space domain that impose 
some degree of isostatic equilibrium to the estimated models 
\citep[e.g.,][]{sunkel1985, sjoberg2009, bagherbandi-eshagh2012, sampietro2015}
or analyze their deviations from a perfect isostatic equilibrium
\citep[e.g.,][]{shin-etal2009}.
\citet{salem-etal2014} presented one of the few local scale 
methods in space domain that simultaneously estimates the basement and Moho
reliefs by imposing isostatic equilibrium. 
They imposed a perfect isostatic equilibrium according to
the Airy's local compensation model \citep{turcotte-schubert2002}, 
which describes well the transition
from continental to oceanic crust at rifted margins 
\citep{worzel1968, watts-moore2017}.

Here, we present a new local scale method for simultaneously estimating the geometries
of basement and Moho along a profile on a passive rifted margin.
Our method is formulated, in space domain, as a non-linear gravity inversion.
In order to produce stable solutions and introduce priori information,
we use different constraints imposing smoothness and lower/upper 
bounds on basement and Moho depths, as well as proximity between estimated 
and known depths at some points along the profile. We also use a constraint
on the lithostatic stress exerted by the interpretation model on a planar surface 
located at depth, below which we assume that there is no lateral density variations.
This constraint, which we conveniently call as \textit{isostatic constraint},
imposes that the lithostatic stress must be smooth along the entire profile,
except at some isolated regions, were it can present abrupt variations.
At these regions, our method enables the estimated model deviates from the 
isostatic equilibrium.
Consequently, our method does not estimate a model in perfect isostatic equilibrium, 
but a model as close as possible to the isostatic equilibrium according to the
Airy's local model. 
Tests with synthetic data show the good performance of our method in simultaneously
retrieving the geometry of basement and Moho of a simulated vulcanic margin
model. Finally, we illustrate an application of our method to invert 
gravity data provided by the global model EIGEN6C4 \citep{forste2014} 
on a profile over the Pelotas basin \citep{stica-etal2014}. This basin is
located at the southern of Brazil and is considered a classical example 
of volcanic passive margin \citep{geoffroy2005}. 
We obtained results that are very consistent with a previous
interpretation presented by \citet{zalan2015}, which used ultra-deep seismic data.
These results show that our method can be very effective at regions
presenting abrupt crustal thinning, which is typically shown in volcanic passive margins.


\section{Methodology}


\subsection{Forward problem}


Let $\mathbf{d}^{o}$ be the observed data vector, whose $i$-th element $d^{o}_{i}$, 
$i = 1, \dots, N$, represent the observed gravity disturbance at the point 
$(x_{i}, y_{i}, z_{i})$, on a profile located over a rifted passive margin. The
coordinates are referred to a topocentric Cartesian system, with $z$ axis pointing
downward, $y$-axis along the profile and $x$-axis perpendicular to the profile. 
We assume that the actual mass distribution in a rifted passive margin can be 
schematically represented according to Figure~\ref{fig:rifted-margin-model}. 
In this model, the subsurface is formed by four layers. 
The first and shallowest one represents a water layer with constant density
$\rho^{(w)}$. 
The second layer is formed by $Q$ vertically adjacent parts representing sediments,
salt or volcanic rocks.
In our example, this layer is formed by two parts with constant densities
$\rho^{(q)}$, $q = 1, 2$. Different models can be created by changing the number $Q$.
The third layer of our model represents the crust. For simplicity, we presume that the
crust density may be equal to $\rho^{(cc)}$, which
represents the continental crust, or equal to $\rho^{(oc)}$, which represents the
oceanic crust.
The deepest layer represents a homogeneous mantle with constant density $\rho^{(m)}$. 
The interface separating the second and third layers defines the basement relief whereas
that separating the third and fourth layers defines the Moho. These surfaces are
represented by dashed-white lines in Figure~\ref{fig:rifted-margin-model}.
We also presume the existence of an isostatic compensation depth at $S_{0}$ 
(represented as a continuous white line in Figure~\ref{fig:rifted-margin-model}),
below which there is no lateral variations in the mass distribution.

In order to define the anomalous mass distribution producing the observed gravity
disturbance, we presume a reference mass distribution formed by two 
layers (not shown). The shallowest layer represents a homogeneous crust with constant
density $\rho^{(r)}$.
The deepest layer in the reference mass distribution represents a homogeneous mantle
with constant density $\rho^{(m)}$. Notice that the mantle in the reference mass 
distribution has the same density as the mantle in our rifted margin model
(Figure~\ref{fig:rifted-margin-model}).
The interface separating the crust and mantle in the reference mass distribution 
is conveniently called \textit{reference Moho} (represented as a continuous white line
in Figure~\ref{fig:rifted-margin-model}).
The reference model can be thought of as the outer layers of a concentric
mass distribution producing the normal gravity field.

We consider that the anomalous mass distribution producing the observed data
is defined as the difference between the rifted margin model 
(Figure~\ref{fig:rifted-margin-model}) and the reference mass distribution (not shown).
As a consequence, the anomalous mass distribution is characterized by regions
with constant density contrast.
This anomalous distribution is approximated by an interpretation model 
formed by $N$ columns of vertically stacked prisms 
(Figure~\ref{fig:interpretation-model}).
For convenience, we presume that there is an observed gravity disturbance over the
center of each column.
We consider that the prisms in the extremities of the interpretation model extend to
infinity along the $y$ axis in order to prevent edge effects in the forward 
calculations. 
The $i$-th column is formed by four vertically adjacent layers, which in turn are
composed of vertically adjacent prisms having infinite length along the $x$-axis.

The first and shallowest layer represents water, 
is formed by a single prism, has thickness $t^{(w)}_{i}$ and a constant density
contrast $\Delta \rho^{(w)} = \rho^{(w)} - \rho^{(r)}$.
The second layer forming the $i$-th column of the interpretation model is defined by
the interpreter, according to the geological environment to be studied and the a priori
information availability. 
As a general rule, this layer can be defined by a set of $Q$ vertically adjacent
prisms, each one with thickness $t^{(q)}_{i}$ and constant density contrast
$\Delta \rho^{(q)} = \rho^{(q)} - \rho^{(r)}$, $q = 1, \dots Q$.
The third layer represents the crust, it is also formed by a single prism,
has thickness $t^{(c)}_{i}$ and density contrast 
$\Delta \rho^{(c)}_{i} = \rho^{(c)} - \rho^{(r)}$, 
with $\rho^{c}$ being the crust density. 
According to our rifted margin model (Figure~\ref{fig:rifted-margin-model}), the crust
density $\rho^{(c)}_{i}$ may assume two possible values, depending on its position
with respect to the $y_{COT}$ (Figure~\ref{fig:interpretation-model}).
As a consequence, the prisms forming the third layer of the interpretation model may
have two possible density contrasts: $\Delta \rho^{(c)}_{i} = \rho^{(cc)} - \rho^{(r)}$,
for $y_{i} \le y_{COT}$, or $\Delta \rho^{(c)}_{i} = \rho^{(oc)} - \rho^{(r)}$,
for $y_{i} > y_{COT}$. 
The top of this layer defines the basement relief and its bottom the relief of the
Moho. 
The fourth layer represents the mantle, it is divided into two parts, each one formed
by a single prism having a constant density contrast 
$\Delta \rho^{(m)} = \rho^{(m)} - \rho^{(r)}$. The shallowest
portion of this layer has thickness $t^{(m)}_{i}$. Its top and bottom define,
respectively, the depths of Moho and the planar isostatic compensation layer $S_{0}$.
The deepest portion of the fourth layer has thickness $\Delta S_{0}$, top at the
surface $S_{0}$ and bottom at the planar surface 
$S_{0} + \Delta S_{0}$, which defines the reference Moho. 

Given the density contrasts, the COT position $y_{COT}$, the isostatic compensation
depth $S_{0}$, the thickness of the water layer and of the $Q-1$ prisms forming the
shallowest portion of the second layer, it is possible to describe the interpretation
model in terms of an $M \times 1$ parameter vector $\mathbf{p}$, $M = 2N + 1$, defined
as follows:
\begin{equation}
\mathbf{p} = \begin{bmatrix}
\mathbf{t}^{Q} \\
\mathbf{t}^{m} \\
\Delta S_{0}
\end{bmatrix} \: ,
\label{eq:parameter-vector}
\end{equation}
where $\mathbf{t}^{Q}$ and $\mathbf{t}^{m}$ are $N \times 1$ vectors whose $i$-th
elements $t^{Q}_{i}$ and $t^{m}_{i}$ represent, respectively, the thickness of the
prism forming the deepest portion of the second layer and the thickness of the prism
forming the shallowest portion of the fourth layer of the interpretation model.
As a consequence, $t^{Q}_{i}$ and $t^{m}_{i}$ approximate, respectively, the geometry
of basement relief and Moho.
In this case, the gravity disturbance produced by the interpretation model (the
predicted gravity disturbance) at the position $(x_{i}, y_{i}, z_{i})$ can be written
as the sum of the vertical component of the gravitational attraction exerted by the $L$
prisms forming the interpretation model as follows:
\begin{equation}
d_{i}(\mathbf{p}) = k_{g} \, G \, \sum_{j = 1}^{L} f_{ij}(\mathbf{p}) \: ,
\label{eq:ith-predicted-data}
\end{equation}
where $f_{ij}(\mathbf{p})$ represents an integral over the volume of the $j$-th 
prism. Here, these volume integrals are computed with the expressions proposed 
by \citet{nagy-etal2000}, by using the open-source Python package 
\textit{Fatiando a Terra} \citep{uieda-etal2013}.


\subsection{Inverse problem formulation}


Let $\mathbf{d}(\mathbf{p})$ be the predicted data vector, whose $i$-th element
$d_{i}(\mathbf{p})$ is the vertical component of the gravitational attraction
(equation~\ref{eq:ith-predicted-data}) exerted by the interpretation model
at the position $(x_{i}, y_{i}, z_{i})$ on the profile.
Estimating the particular parameter vector producing a predicted data
$\mathbf{d}(\mathbf{p})$ as close as possible to the observed data 
$\mathbf{d}^{o}$ can be formulated as the problem of minimizing the goal function
\begin{equation}
\Gamma (\mathbf{p}) = \Phi(\mathbf{p}) + \sum_{\ell = 0}^{3} \alpha_{\ell}
\Psi_{\ell}(\mathbf{p}) \: ,
\label{eq:goal-function}
\end{equation}
subject to the inequality constraint 
\begin{equation}
p_{j}^{min} < p_{j} < p_{j}^{max} \: , \quad j = 1, \dots, M \: ,
\label{eq:inequality-constraint}
\end{equation}
where $p_{j}^{min}$ and $p_{j}^{max}$ define, respectively, lower and upper bounds 
for the $j$-th element of $\mathbf{p}$.
In equation~\ref{eq:goal-function}, $\alpha_{\ell}$ is the weight assigned to the 
$\ell$-th regularizing function $\Psi_{\ell}(\mathbf{p})$, $\ell = 0, 1, 2, 3$, and
$\Phi(\mathbf{p})$ is the misfit function given by
\begin{equation}
\Phi(\mathbf{p}) = \frac{1}{N} \| \mathbf{d}^{o} - \mathbf{d}(\mathbf{p}) \|_{2}^{2} 
\: , 
\label{eq:misfit-function}
\end{equation}
where $\| \cdot \|_{2}^{2}$ represents the squared Euclidean norm. Details about 
the regularizing functions $\Psi_{\ell}(\mathbf{p})$, $\ell = 0, 1, 2, 3$ and the 
numerical procedure to solve this non-linear inverse problem are given in the following 
sections.


\subsection{Isostatic constraint}

Consider that no vertical forces are acting on the lateral surfaces of 
each column forming the model (Figure \ref{fig:interpretation-model}). In this case, 
the lithostatic stress (pressure) exerted by the $i$-th column at the surface $S_{0}$
can be computed as follows \citep{turcotte-schubert2002}:
\begin{equation}
t^{(w)}_{i} \rho^{(w)} + t^{(1)}_{i} \rho^{(1)}_{i} + \dots + 
t^{(Q)}_{i} \rho^{(Q)}_{i} + t^{(c)}_{i} \rho^{(c)}_{i} + t^{(m)}_{i} \rho^{(m)} 
= \tau_{i} \: ,
\label{eq:lithostatic-stress-densities}
\end{equation}
where $\tau_{i}$ is the ratio of lithostatic stress to the mean gravity 
value on the study area.
By rearranging terms in equation~\ref{eq:lithostatic-stress-densities} 
and using the relation
\begin{equation}
S_{0} = t^{(w)}_{i} + t^{(1)}_{i} + \dots + t^{(Q)}_{i} + t^{(c)}_{i} + t^{(m)}_{i} \: ,
\label{eq:S0}
\end{equation}
it is possible to show that:
\begin{equation}
\Delta \tilde{\rho}^{(Q)}_{i} \, t^{(Q)}_{i} + 
\Delta \tilde{\rho}^{(m)}_{i} \, t^{(m)}_{i} + 
\Delta \tilde{\rho}^{(w)}_{i} \, t^{(w)}_{i} + 
\Delta \tilde{\rho}^{(1)}_{i} \, t^{(1)}_{i} +
\dots + 
\Delta \tilde{\rho}^{(Q-1)}_{i} \, t^{(Q-1)}_{i} +
\rho^{(c)}_{i} \, S_{0} = \tau_{i} \: ,
\label{eq:lithostatic-stress-density-contrasts}
\end{equation}
where $\Delta \tilde{\rho}^{(\alpha)}_{i} = \rho^{(\alpha)}_{i} - \rho^{(c)}_{i}$, 
$\alpha = w, 1, \dots, Q-1, Q, m$.
In order to describe the lithostatic stress exerted by all columns forming the
interpretation model on the surface $S_{0}$, 
equation~\ref{eq:lithostatic-stress-density-contrasts} 
can be written as follows:
\begin{equation}
\mathbf{M}^{(Q)} \mathbf{t}^{(Q)} + \mathbf{M}^{(m)} \mathbf{t}^{(m)} + \mathbf{M}^{(w)}
\mathbf{t}^{(w)} + \mathbf{M}^{(1)} \mathbf{t}^{(1)} + \dots + \mathbf{M}^{(Q-1)}
\mathbf{t}^{(Q-1)} + \boldsymbol{\rho}^{(c)} S_{0} = \boldsymbol{\tau} \: ,
\label{eq:lithostatic-stress-matrix}
\end{equation}
where $\boldsymbol{\tau}$ is an $N \times 1$ vector whose $i$-th element is
the $\tau_{i}$ (equation~\ref{eq:lithostatic-stress-densities}) associated with
the $i$-th column, $\mathbf{t}^{(\alpha)}$, $\alpha = w, 1, \dots, Q-1, Q, m$, 
is a $N \times 1$ vector with $i$-th element defined by the thickness 
$t^{(\alpha)}_{i}$ of a prism in the $i$-th column, $\mathbf{M}^{(\alpha)}$ is an $N \times N$ diagonal matrix whose elements in the main diagonal are defined by the density contrasts 
$\Delta \tilde{\rho}^{(\alpha)}_{i}$, $i = 1, \dots, N$, of the prisms in a layer and
$\boldsymbol{\rho}^{(c)}$ is an $N \times 1$ vector containing the densities of the prisms
representing the crust. 

Let us now consider that the interpretation model is in isostatic equilibrium
according to the Airy's local model \citep[e.g.,][]{turcotte-schubert2002,
hofmann-wellenhof-moritz2005, lowrie2007}. We impose this condition 
by applying the first-order Tikhonov regularization
\citep{aster-etal2005} to the vector $\boldsymbol{\tau}$
(equation~\ref{eq:lithostatic-stress-matrix}), obtaining the
following expression:
\begin{equation}
\mathbf{R} \left( \mathbf{C} \mathbf{p} + \mathbf{D} \mathbf{t} \right) = \mathbf{0} 
\: ,
\label{eq:tik1-lithostatic-stress}
\end{equation}
where $\mathbf{0}$ is a vector with null elements and the remaining terms are given by:
\begin{equation}
\mathbf{C} = \begin{bmatrix}
\mathbf{M}^{(Q)} & \mathbf{M}^{(m)} & \mathbf{0}
\end{bmatrix}_{N \times M} \: ,
\label{eq:matrix-C}
\end{equation}
\begin{equation}
\mathbf{D} = \begin{bmatrix}
\mathbf{M}^{(w)} & \mathbf{M}^{(1)} & \cdots & \mathbf{M}^{(Q-1)} &
\boldsymbol{\rho}^{(c)}
\end{bmatrix}_{N \times \left( QN + 1 \right)} \: ,
\label{eq:matrix-D}
\end{equation}
\begin{equation}
\mathbf{t} = \begin{bmatrix}
\mathbf{t}^{(w)} \\ \mathbf{t}^{(1)} \\ \vdots \\ \mathbf{t}^{(Q-1)} \\ S_{0}
\end{bmatrix}_{\left( QN + 1 \right) \times 1}\: ,
\label{eq:vector-t}
\end{equation}
and $\mathbf{R}$ is an $\left( N-1 \right) \times N$ matrix, whose element 
$ij$ is defined as follows:
\begin{equation}
\left[ \mathbf{R} \right]_{ij} = \begin{cases}
1 &, \quad j = i \\
-1 &, \quad j = i + 1 \\
0 &, \quad \text{otherwise}
\end{cases} \quad .
\label{eq:matrix-R}
\end{equation}
Finally, from equation~\ref{eq:tik1-lithostatic-stress}, it is possible to
define the regularizing function $\Psi_{0}(\mathbf{p})$ 
(equation~\ref{eq:goal-function}): 
\begin{equation}
\Psi_{0}(\mathbf{p}) = \| \mathbf{W} \, \mathbf{R} \left( \mathbf{C} \mathbf{p} + \mathbf{D}
\mathbf{t} \right) \|_{2}^{2} \: ,
\label{eq:isostatic-constraint-function}
\end{equation}
where $\mathbf{W}$ is an $(N - 1) \times (N - 1)$ diagonal matrix having constant
elements $0 < w_{ii} \le 1$, $i = 1, \dots, N - 1$. 
Function $\Psi_{0}(\mathbf{p})$ defines the \textit{Isostatic constraint}.

Notice that, by minimizing the function $\Psi_{0}(\mathbf{p})$ 
(equation~\ref{eq:isostatic-constraint-function}), the method imposes smoothness on 
the lithostatic stress exerted by the interpretation model on the isostatic 
compensation surface $S_{0}$.
Matrix $\mathbf{W}$ controls the relative amount of isostatic equilibrium imposed 
along the profile. 
In the particular case in which all diagonal elements $w_{ii}$ have the same
constant value, the same amount of isostatic equilibrium is imposed along the 
whole profile. On the other hand, different amounts of isostatic equilibrium
can be imposed along the profile by varying the values of these elements.
Elements $w_{ii} \approx 1$ impose a smooth lithostatic stress curve 
at the transition between columns $i$ and $i+1$ of the interpretation model.
Elements $w_{ii} \approx 0$ allow abrupt variations in the lithostatic stress 
curve between columns $i$ and $i+1$ of the interpretation model.
By using all elements $w_{ii} = 1$, we impose full isostatic equilibrium 
along the entire profile. Alternatively, we may enable the interpretation model 
deviates from the isostatic equilibrium by conveniently decreasing the numerical values
assigned to the elements $w_{ii}$ at specific regions along the profile. 
The strategy used to define the elements $w_{ii}$ is 
presented in the specific section describing the numerical solution of the inverse problem.


\subsection{Smoothness constraint}


This constraint imposes smoothness on the adjacent thickness of the prisms forming the
deepest portion of the second layer and the shallowest part of the fourth layer
of the interpretation model by applying the first-order Tikhonov regularization
\citep{aster-etal2005} to the vectors $\mathbf{t}^{(Q)}$ and $\mathbf{t}^{(m)}$ 
(equation~\ref{eq:parameter-vector}). 
Mathematically, this constraint is represented by the regularizing function
$\Psi_{1}(\mathbf{p})$ (equation~\ref{eq:goal-function}):
\begin{equation}
\Psi_{1}(\mathbf{p}) = \| \mathbf{S}\mathbf{p} \|_{2}^{2} \: ,
\label{eq:smootheness-contraint}
\end{equation}
where $\mathbf{S}$ is an $2 \left( N-1 \right) \times M$ matrix given by:
\begin{equation}
\mathbf{S} = \begin{bmatrix}
\mathbf{R} & \mathbf{0} & \mathbf{0} \\
\mathbf{0} & \mathbf{R} & \mathbf{0}
\end{bmatrix} \: ,
\label{eq:matrix-S}
\end{equation}
where $\mathbf{R}$ is defined by equation~\ref{eq:matrix-R} and $\mathbf{0}$
are matrices with all elements equal to zero.


\subsection{Equality constraint}

\subsubsection*{Equality constraint on vector $\mathbf{t}^{Q}$}

Let $\mathbf{a}$ be a vector whose $k$-th element $a_{k}$,
$k = 1, \dots, A$, is the difference between a known basement depth and
the sum of the thickness of the upper portions of the interpretation model
(water layer and the upper parts of second layer), all at the same
horizontal coordinate $y^{A}_{k}$ of the profile. 
These differences, which must be positive, are used to define 
the regularizing function $\Psi_{2}(\mathbf{p})$ 
(equation~\ref{eq:goal-function}):
\begin{equation}
\Psi_{2}(\mathbf{p}) = \| \mathbf{A}\mathbf{p} - \mathbf{a} \|_{2}^{2} \: ,
\label{eq:equality-constraint-basement}
\end{equation}
where $\mathbf{A}$ is an $A \times M$ matrix whose $k$-th line has one element 
equal to one and all the remaining elements equal to zero. The location of the
single non-null element in the $k$-th line of $\mathbf{A}$ depends on the coordinate
$y^{A}_{k}$ of the known basement depth $a_{k}$. Let us consider, 
for example, an interpretation model formed by $N = 10$ columns. Consider also that 
the thickness of the deepest part $Q$ of the second layer forming the interpretation model 
at the coordinates $y^{A}_{1} = y_{4}$ and $y^{A}_{2} = y_{9}$ 
are equal to $25$ and $35.7$ km, respectively. In this case, $A = 2$,
$\mathbf{a}$ is a $2 \times 1$ vector with elements $a_{1} = 25$ and $a_{2} = 35.7$
and $\mathbf{A}$ is a $2 \times M$ matrix ($M = 2N + 1 = 21$). The element $4$ of the
first line and the element $9$ of the second line of $\mathbf{A}$ are equal to $1$ and
all its remaining elements are equal to zero.

\subsubsection*{Equality constraint on vector $\mathbf{t}^{m}$}

Let $\mathbf{b}$ be a vector whose $k$-th element $b_{k}$,
$k = 1, \dots, B$, is the difference between the isostatic compensation depth
$S_{0}$ and the known Moho depth at the horizontal coordinate $y^{B}_{k}$ of the
profile. These differences, which must be positive, define known thickness values
of the upper part of the fourth layer forming the interpretation model.
These values are used to define the 
regularizing function $\Psi_{3}(\mathbf{p})$ (equation~\ref{eq:goal-function}):
\begin{equation}
\Psi_{3}(\mathbf{p}) = \| \mathbf{B}\mathbf{p} - \mathbf{b} \|_{2}^{2} \: ,
\label{eq:equality-constraint-moho}
\end{equation}
where $\mathbf{B}$ is a $B \times M$ matrix whose $k$-th line has one element 
equal to one and all the remaining elements equal to zero. This matrix is defined 
in the same way as matrix $\mathbf{A}$ (equation~\ref{eq:equality-constraint-basement}).


\subsection{Computational procedures for solving of the inverse problem}

The parameter vector $\mathbf{p}$ (equation~\ref{eq:parameter-vector}) minimizing the
goal function $\Gamma (\mathbf{p})$ (equation~\ref{eq:goal-function}), subjected to
the inequality constraint (equation~\ref{eq:inequality-constraint}), is estimated 
in three steps. At each step, the goal function is minimized by using the Levenberg-Marquardt 
method \citep{aster-etal2005} and the inequality constraint 
(equation~\ref{eq:inequality-constraint}) is incorporated by using the same strategy 
employed by \citet{barbosa-etal1999}.
All derivatives of the misfit function $\Phi(\mathbf{p})$ (equation~\ref{eq:misfit-function})
with respect to the parameters are computed by using a finite difference approximation.

The first step consists in solving the inverse problem without imposing the isostatic
constraint, by using $\alpha_{0} = 0$ (equation \ref{eq:goal-function}). For this step,
the interpreter must set:
\begin{itemize}
	\item \underline{Parameters defining the interpretation model
	(Figure~\ref{fig:interpretation-model}):} density contrasts $\Delta \rho^{(\alpha)}$, 
	$\alpha = w, 1, \dots, Q, cc, oc, m$, of the four layers, COT position $y_{COT}$ and
	isostatic compensation depth $S_{0}$. Figure \ref{fig:interpretation-model} illustrates 
	the case in which the second layer is formed by $Q = 2$ parts. This number, however, 
	can be changed according to the study area.
	\item \underline{Parameters for the inversion:} weights $\alpha_{\ell}$,
	$\ell = 1, 2, 3$ (equation~\ref{eq:goal-function}), associated to the 
	smoothness and equality constraints (equations \ref{eq:smootheness-contraint},
	\ref{eq:equality-constraint-basement} and \ref{eq:equality-constraint-moho}),
	lower and upper bounds $p_{j}^{min}$ and $p_{j}^{max}$
	(equation~\ref{eq:inequality-constraint}), $j = 1, \dots, M$,
	for the parameters to be estimated, vectors $\mathbf{a}$ 
	(equation~\ref{eq:equality-constraint-basement}) and $\mathbf{b}$
	(equation~\ref{eq:equality-constraint-moho}) containing known thickness values
	and an initial approximation $\mathbf{p}^{(0)}$ for the parameter vector $\mathbf{p}$
	(equation~\ref{eq:parameter-vector}). The initial approximation $\mathbf{p}^{(0)}$
	must satisfy the inequality constraints (equation~\ref{eq:inequality-constraint}).
\end{itemize}
The estimated parameter vector obtained at the end of this first step is conveniently called
$\mathbf{p}^{(1)}$. The main goal in this step is finding suitable values for the parameters
defining the interpretation model and those used for inversion. Several trials may be
necessary to find suitable values for these parameters. 

The second step consists in obtaining an estimated parameter vector $\mathbf{p}^{(2)}$
by imposing full isostatic equilibrium on the interpretation model along the entire profile.
In this step, the interpreter must use the same initial approximation $\mathbf{p}^{(0)}$
for the parameter vector $\mathbf{p}$ (equation~\ref{eq:parameter-vector}). 
Additionally, the interpreter must find a suitable 
value for the weight $\alpha_{0} = 0$ (equation \ref{eq:goal-function}) controlling the 
isostatic constraint, by using the matrix $\mathbf{W}$
(equation~\ref{eq:isostatic-constraint-function}) equal to the identity.
We presume that, by imposing full isostatic equilibrium along the entire profile,
the estimated parameter vector $\mathbf{p}^{(2)}$ will produce a good data fit,
except at some isolated regions, where there will be large residuals between the
observed and predicted data. We assume that, at these regions, the study area deviates
from the isostatic equilibrium described by the Airy's local model.

Finally, the third step consists in obtaining an estimated parameter vector $\mathbf{p}^{(3)}$
by imposing different amounts of isostatic equilibrium on the interpretation model along the
entire profile. At this step, the interpreter must compute the diagonal elements 
$w_{ii}$ of the matrix $\mathbf{W}$
(equation~\ref{eq:isostatic-constraint-function}) that enable the interpretation 
model deviates from the isostatic equilibrium at the regions presenting 
large residuals. The elements of $\mathbf{W}$ are computed as follows:
\begin{equation} \label{eq:elements-wii}
w_{ii} = 
\exp \left[ - \frac{ \left( r_{i}^{(2)} + r_{i+1}^{(2)} \right)^{2}}{4 \sigma} \right] \: ,
\end{equation}
where $\sigma$ is a positive constant, $\mathbf{p}^{(2)}$ is the estimate parameter vector
obtained in the previous step and the variables
$r_{i}^{(2)} = d^{o}_{i} - d_{i} \left( \mathbf{p}^{(2)} \right)$ and 
$r_{i+1}^{(2)} = d^{o}_{i+1} - d_{i+1} \left(\mathbf{p}^{(2)} \right)$ represent,
respectively, the residuals between observed and predicted data
(equation~\ref{eq:ith-predicted-data}) at the positions $(x_{i}, y_{i}, z_{i})$ and 
$(x_{i+1}, y_{i+1}, z_{i+1})$. Equation~\ref{eq:elements-wii} defines elements 
$w_{ii}$ in the interval $\left] 0, 1 \right]$. Additionally, it results in 
$w_{ii} \approx 1$ at regions where the residuals are close to zero and 
$w_{ii} \approx 0$ at regions presenting large residuals.
The positive constant $\sigma$ controls the deviation from isostatic equilibrium.
Large $\sigma$ values allow small deviations from isostatic equilibrium at
regions presenting large residuals.
This strategy to define the elements of matrix $\mathbf{W}$ 
(equation\ref{eq:isostatic-constraint-function}) presumes that the isostatic constraint may
produce large residuals at some regions along the profile. To counteract this problem,
our method enables the interpretation model deviates from isostatic equilibrium at these 
regions. This idea is in perfect agreement with the fact that the isostatic equilibrium
in a continental margin cannot be perfectly explained by using the Airy's local model.

Other important aspect of our method is related to the values attributed to 
the weights $\alpha_{\ell}$ (equation~\ref{eq:goal-function}).
Their values can be very dependent on the particular characteristics of the 
interpretation model and there is no analytical rule to define them. 
To overcome this problem, we normalize the $\alpha_{\ell}$ values as follows:
\begin{equation}
\alpha_{\ell} = \tilde{\alpha}_{\ell} \, \frac{E_{\Phi}}{E_{\ell}} \: 
, \quad \ell = 0, 1, 2, 3 \: ,
\label{eq:constraint-weights}
\end{equation}
where $\tilde{\alpha}_{\ell}$ is a positive scalar and $E_{\Phi} / E_{\ell}$
is a normalizing constant. In this equation, $E_{\ell}$ represents the median 
of the elements forming the main diagonal of the Hessian matrix of the 
$\ell$-th constraining function $\Psi_{\ell}(\mathbf{p})$ 
(equations~\ref{eq:isostatic-constraint-function}, 
\ref{eq:smootheness-contraint}, \ref{eq:equality-constraint-basement} and 
\ref{eq:equality-constraint-moho}). The constant $E_{\Phi}$ is defined in 
a similar way by using the Hessian matrix of the misfit function $\Phi(\mathbf{p})$ 
(equation~\ref{eq:misfit-function}) computed with 
the initial approximation $\mathbf{p}^{(0)}$ for the parameter vector $\mathbf{p}$ 
(equation~\ref{eq:parameter-vector}) at the first step.
According to this empirical strategy, the weights $\alpha_{\ell}$ are defined 
by using the positive scalars $\tilde{\alpha}_{\ell}$ 
(equation \ref{eq:constraint-weights}), which are less dependent on the particular 
characteristics of the interpretation model.

The algorithm can be summarized as follows:

\begin{enumerate}
	\item [\textbf{(Step 1)}] Use $\tilde{\alpha}_{0} = 0$ and set non-null values for $\tilde{\alpha}_{1}$, $\tilde{\alpha}_{2}$ and $\tilde{\alpha}_{3}$ (equation~\ref{eq:constraint-weights}).
	Define $p_{j}^{min}$, $p_{j}^{max}$ (equation~\ref{eq:inequality-constraint}),
	$\mathbf{a}$ (equation~\ref{eq:equality-constraint-basement}) and
	$\mathbf{b}$ (equation~\ref{eq:equality-constraint-moho}).
	Define an initial approximation $\mathbf{p}^{(0)}$ satisfying the inequality constraint
	(equation~\ref{eq:inequality-constraint}). Use $\mathbf{p}^{(0)}$ to compute the 
	Hessian matrix of the misfit function $\Phi(\mathbf{p})$ (equation~\ref{eq:misfit-function}).
	Compute the Hessian matrices of the constraining functions $\Psi_{\ell}(\mathbf{p})$, 
	$\ell = 1, 2, 3$ (equations \ref{eq:smootheness-contraint},
	\ref{eq:equality-constraint-basement} and \ref{eq:equality-constraint-moho}). 
	Compute $E_{\Phi}$, $E_{\ell}$ and $\alpha_{\ell}$
	(equation~\ref{eq:constraint-weights}), $\ell = 1, 2, 3$. 
	Estimate a parameter vector $\mathbf{p}^{(1)}$
	minimizing $\Gamma(\mathbf{p})$ (equation~\ref{eq:goal-function}), 
	subject to the inequality constraint (equation~\ref{eq:inequality-constraint}).
	\item [\textbf{(Step 2)}] Use the same initial approximation $\mathbf{p}^{(0)}$
	of the previous step. Set $\mathbf{W}$ (equation~\ref{eq:isostatic-constraint-function})
	equal to identity. Set a non-null value for $\tilde{\alpha}_{0}$
	(equation~\ref{eq:constraint-weights}).
	Compute the Hessian matrix of the constraining function $\Psi_{0}(\mathbf{p})$
	(equation~\ref{eq:isostatic-constraint-function}).
	Compute $E_{0}$ and $\alpha_{0}$ (equation~\ref{eq:constraint-weights}).
	Estimate a parameter vector $\mathbf{p}^{(2)}$ minimizing $\Gamma(\mathbf{p})$ (equation~\ref{eq:goal-function}), subject to the inequality constraint 
	(equation~\ref{eq:inequality-constraint}).
	\item [\textbf{(Step 3)}] Use $\mathbf{p}^{(2)}$ as initial approximation.
	Set the positive constant $\sigma$ and compute the 
	diagonal elements $w_{ii}$ (equation~\ref{eq:elements-wii}), $i = 1, \dots N-1$.
	With the new $\mathbf{W}$, compute the Hessian matrix of the constraining function
	$\Psi_{0}(\mathbf{p})$ (equation~\ref{eq:isostatic-constraint-function}).
	Estimate a parameter vector $\mathbf{p}^{(3)}$ minimizing $\Gamma(\mathbf{p})$ (equation~\ref{eq:goal-function}), subject to the inequality constraint 
	(equation~\ref{eq:inequality-constraint}).
\end{enumerate}


\section{Applications to synthetic data}


We have simulated a simple volcanic margin model formed by four layers: 
water, sediments + seaward dipping reflectors, crust (continental and oceanic) 
and mantle. 
Parameters defining this model are shown in Table~\ref{tab:volcanic-margin-model}.
By following the algorithm described in the previous section, we applied our method 
to invert the synthetic gravity disturbance data produced by our volcanic margin model.

Figure \ref{fig:volc-margem-step1} shows the results obtained at Step 1 of our
algorithm.
The interpretation model was defined by using the parameters shown in
Table~\ref{tab:volcanic-margin-model}.
The parameters $\tilde{\alpha}_{1}$, $\tilde{\alpha}_{2}$ and 
$\tilde{\alpha}_{3}$ (equation~\ref{eq:constraint-weights}) used to estimate a parameter 
vector $\mathbf{p}^{(1)}$ without imposing the isostatic constraint have values $10^{1}$, $10^{1}$ and $10^{2}$, respectively.
As we can see, the estimated model produces a gravity disturbance and a 
lithostatic stress very close to the simulated ones and the
estimated Moho retrieves the true one.
On the other hand, the estimated basement relief is very smooth and does not
retrieve the true one along the first $200$ km of the profile.

Figure~\ref{fig:volc-margem-step2} shows the estimated model obtained at the end
of Step 2, by using $\tilde{\alpha}_{0} = 10^{2}$ (equation~\ref{eq:constraint-weights}).
In comparison to the estimated model obtained in the Step 1 (Figure~\ref{fig:volc-margem-step1}),
this result shows a very smooth lithostatic stress curve as a consequence of the
isostatic constraint. The use of the isostatic constraint has produced a worse estimated Moho, 
but has improved the estimated basement relief. The main improvement occurs along the 
first $\approx 100$ km on the profile, where the true model exhibits a pronounced crustal
thinning. The region between approximately $100$ and $200$ km, however, shows large 
differences between the estimated and true basement reliefs. At this region, we can also
notice the presence of large differences between the simulated and predicted gravity
disturbances.

Figure~\ref{fig:volc-margem-step3} shows the estimated model obtained at the Step 3
of our algorithm, by using $\sigma = 21$ (equation~\ref{eq:elements-wii}). 
In comparison with the estimated model obtained at Step 2 
(Figure~\ref{fig:volc-margem-step2}), the new estimated model produces a very good 
data fit along the whole profile and shows an estimated Moho closer to the true one.
The estimated basement relief is worse than that obtained at Step 2
(Figure~\ref{fig:volc-margem-step2}), but it is still much better than that obtained
at Step 1 (Figure~\ref{fig:volc-margem-step1}).
It worth noting that the new lithostatic stress curve contains an abrupt variation
at the region located between $100$ and $200$ km,
showing that our method was able to successfully enables
the interpretation model deviates from the isostatic equilibrium at an
isolated region. This region coincides with that presenting the large absolute 
differences between the gravity disturbance data in Figure~\ref{fig:volc-margem-step2}.

The estimated model obtained at Step 3 (Figure~\ref{fig:volc-margem-step3})
by using the isostatic constraint is superior than that
obtained at Step 1 (Figure~\ref{fig:volc-margem-step1}),
without using the isostatic constraint.
It is possible to obtain a good estimated model without using the 
isostatic constraint, but it requires a great number of known depths
along the profile. This can be verified by comparing the estimated models
shown in Figures \ref{fig:volc-margem-step1-more-inf} and
\ref{fig:volc-margem-step3-more-inf}. The first one shows an estimated model
obtained in the same way as that shown in 
Figure~\ref{fig:volc-margem-step1}, but by using a great number of additional 
known depths at basement. As we can see, the additional known depths
improve the estimated basement relief.
In practical situations, however, the amount of available priori 
information on the study area is limited.
Figure~\ref{fig:volc-margem-step3-more-inf} shows an estimated model
obtained in the same way as that shown in Figure~\ref{fig:volc-margem-step3},
but by using a limited number of additional known depths at basement.
Notice that the number of additional known depths is smaller than that
used in Figure~\ref{fig:volc-margem-step1-more-inf}.
This result shows that the isostatic constraint can be very effective in 
retrieving the basement relief and Moho by using a limited amount of
priori information. 


\section{Application to real data}

We applied our method to interpret a gravity profile over the Pelotas basin
\citep{stica-etal2014}, located at the southern of Brazil. This basin is 
considered a classical example of volcanic margin \citep{geoffroy2005}.
We apply our method to the gravity disturbance data provided by the EIGEN6C4
model \citep{forste2014} on the study area. 

Figure~\ref{fig:pelotas-step3} shows the estimated model obtained at the Step 3
of our algorithm, by using $\sigma = 58$ (equation~\ref{eq:elements-wii}), 
$\tilde{\alpha}_{0} = 10^{2}$, $\tilde{\alpha}_{1} = 10^{1}$, 
$\tilde{\alpha}_{2} = 10^{1}$ and $\tilde{\alpha}_{3} = 10^{2}$
(equation~\ref{eq:constraint-weights}).
Parameters defining the interpretation model are the same used
to define our volcanic margin model (Table~\ref{tab:volcanic-margin-model}).
The initial approximation used in Step 1 (not shown) was obtained by interpreting a
seismic section presented by \citet{stica-etal2014}.
As we can see, the estimated model produces a very good data fit along the whole
profile. Our estimated model is very close to that obtained independently by
\citet{zalan2015}.
The larger differences ($\approx 10$ km) occur at the basement, along the first 
$100$ km of the profile.
At this region, \citeauthor{zalan2015} proposes a steep variation in basement
relief, which shows a maximum depth $\approx 30$ km. On the other hand, our result 
shows a smooth variation in basement relief. Despite these large differences,
Figure~\ref{fig:pelotas-step3} shows that our model produces a very good data fit
at this region.
It worth noting that the estimated lithostatic stress curve contains abrupt variations
at an isolated region close to $150$ km, indicating deviations of our estimated model
from the isostatic equilibrium predicted by the Airy's local model.
On the remaining parts of the profile, the lithostatic stress curve is very smooth,
suggesting that the Pelotas basin is in isostatic equilibrium according to the
Airy's model on those regions.
Figure~\ref{fig:pelotas-step3-more-infs} shows an estimated model
obtained in the same way as that shown in Figure~\ref{fig:pelotas-step3},
but by using additional known depths at basement.
Similarly to that shown in Figure~\ref{fig:pelotas-step3},
the new model (Figure~~\ref{fig:pelotas-step3-more-infs}) produces a very good data fit 
and deviates from the isostatic equilibrium at the position $\approx 150$ km. 
Notice that, by using the additional known depths, the differences between our model
and that obtained by \citet{zalan2015} decrease considerably without producing
significant changes in the predicted data.
It is worth stressing that our results were obtained by using gravity data
provided by the global gravity model EIGEN6C4 \citep{forste2014}, whereas 
\citet{zalan2015} obtained his result by interpreting ultra-deep seismic sections
available to the petroleum industry.


\section{Conclusions}

We present a new local scale method for simultaneously estimating the geometries of
basement and Moho on a profile located on a passive rifted margin.

Our method approximates the subsurface by a set of adjacent columns, each one divided
into a four layers. The first one represents water; the second can be formed by more than
one part, according to the complexity of the study area; the third represents the crust 
and the last one the mantle.

The interfaces defining the top and bottom of the third layer represent, respectively, 
the basement and Moho.

All layers have a constant density contrast, except the third, which can assume two 
possible values representing the continental and oceanic crusts.

We assume that Crust-Ocean Transition (COT) can be represented by an abrupt and vertical 
interface.

Our method presumes the knowledge of the COT position, all density contrasts 
and the geometry of all layers above the basement.

The method is formulated as a constrained and non-linear inverse problem. 

In order to obtain stable solutions, we force the estimated basement and Moho to be 
smooth and close to some points having known depths along the profile.

We also impose that the lithostatic stress exerted by the model on a planar surface 
located at depth must be mostly smooth, except at some isolated regions, where it can
present abrupt variations.

 

Tests with synthetic data show that the isostatic constraint can considerably improve 
the estimated model at regions showing pronounced crustal thinning, which 
are typical of volcanic passive margins. 

Applications to real data over the Pelotas basin, considered a classical volcanic
margin at the southern of Brazil, produced results in agreement with a previous 
interpretation obtained independently at the study area by using ultra-deep seismic
data.

Further research could be conducted to generalize our method for estimating
three dimensional models.

%%%%%%%%%%%%%%%%%% Tables %%%%%%%%%%%%%%%%%%%%%%%%%%%%%%%%%%%%%%%%%%%%%%%%

\tabl{volcanic-margin-model}{Properties of the volcanic magin model. 
	The model extends from $y = 0$ km to $y = 383$ km, the Crust-Ocean 
	Transition (COT) is located at $y_{COT} = 350$ km and the reference 
	Moho is located at $S_{0} + \Delta S = 43 \, 200$ km, where 
	$\Delta S = 2 \, 200$ km (Figures~\ref{fig:rifted-margin-model} and 
	\ref{fig:interpretation-model}).
	The density contrasts $\Delta\rho^{(\alpha)}$ are defined with respect to the
	reference value $\rho^{(r)} = 2870$ kg/m$^{3}$, which coincides with
	the density $\rho^{(cc)}$ attributed to the continental crust.
	\label{tab:volcanic-margin-model}
}{
\begin{center}
	\begin{tabular}[]{lccc}
		\hline
		\textbf{Geological meaning} & $\rho^{(\alpha)}$ (kg/m$^{3}$) & $\Delta\rho^{(\alpha)}$ (kg/m$^{3}$) & $\alpha$ \\
		\hline
		water & $1030$ & $-1840$ & $w$ \\
		\hline
		sediments & $2350$ & $-520$ & $1$ \\
		SDR & $2855$ & $-15$ & $2$ \\
		\hline 
		continental crust & $2870$ & $0$ & $cc$ \\
		oceanic crust & $2885$ & $15$ & $oc$ \\
		\hline
		mantle & $3240$ & $370$ & $m$ \\
		\hline
	\end{tabular}
\end{center}
}


%%%%%%%%%%%%%%%%%% Tables %%%%%%%%%%%%%%%%%%%%%%%%%%%%%%%%%%%%%%%%%%%%%%%%

%%%%%%%%%%%%%%%%%% Figures %%%%%%%%%%%%%%%%%%%%%%%%%%%%%%%%%%%%%%%%%%%%%%%

%% Methodology

\plot{rifted-margin-model-color}{width=\columnwidth}{
	{Rifted margin model formed by four layers. The first one 
	represents a water layer with constant density $\rho^{(w)}$. The second 
	layer is formed by $Q$ vertically adjacent parts, according to the
	geological are to be studied. In this example, $Q = 2$.
	These parts represent sediments, salt or volcanic rocks and have constant 
	densities $\rho^{(q)}$, $q = 1, \dots, Q$. The
	third layer represents the crust, which is divided into the continental crust, with
	a constant density $\rho^{(cc)}$, and the oceanic crust, with a constant density
	$\rho^{(oc)}$. We presume an abrupt Crust-Ocean Transition (COT). Finally, the
	fourth layer of our model represents a homogeneous mantle with constant density
	$\rho^{(m)}$. Basement and Moho are represented by the dashed-white lines. 
	The continuous white lines represent the isostatic compensation depth
	at $S_{0}$ and the reference Moho at $S_{0} + \Delta S$.}
	\label{fig:rifted-margin-model}
}

\plot{interpretation-model-color}{width=\columnwidth}{
	{Interpretation model formed by $N$ columns of vertically stacked prisms. 
	Each column is formed by four layers of prisms and locally approximates the four
	layers of the rifted margin model (Figure~\ref{fig:rifted-margin-model-color}).
	Each prism has a constant density contrast defined as the difference between
	its corresponding density at the rifted margin model (Figure~\ref{fig:rifted-margin-model-color})
	and the constant density $\rho^{(r)}$ of the shallowest layer forming
	the reference density distribution (see text).
	Basement and Moho are represented by the dashed-white lines. 
	The continuous white line represents the isostatic compensation depth
	at $S_{0}$. The base of the interpretation model coincides with the
	reference Moho located at $S_{0} + \Delta S$.}
	\label{fig:interpretation-model}
}

%% Results

\plot{volcanic-margin-grafics-estimated-model-alphas_X_1_1_2}{width=0.6\textwidth}{
	{Application to synthetic data. Results obtained in Step 1. 
	(Bottom panel) Estimated and true surfaces,
	the initial basement and Moho used in the inversion, 
	as well as the known depths at basement and Moho.
	(Middle panel) True and estimated lithostatic stress curves computed
	by using equation~\ref{eq:lithostatic-stress-densities}. The values are multiplied 
	by a constant gravity value equal to $9.81$ m/s$^{2}$.
	(Upper panel) True gravity disturbance data produced by the volcanic 
	margin model (simulated data), data produced by the estimated model 
	(predicted data) and data produced the by model used as initial guess in the 
	inversion (initial guess data).
	The contour of the prisms forming the interpretation model were omitted.
	The density contrasts were defined according to Table~\ref{tab:volcanic-margin-model}.}
	\label{fig:volc-margem-step1}
}

\plot{volcanic-margin-grafics-estimated-model-alphas_2_1_1_2}{width=0.6\textwidth}{
	{Application to synthetic data. Results obtained in Step 2.
	The remaining informations are the same shown in the caption of
	Figure~\ref{fig:volc-margem-step1}.
	\label{fig:volc-margem-step2}}
}

\plot{volcanic-margin-grafics-estimated-model-alphas_2_1_1_2-sgm_21}{width=0.6\textwidth}{
	{Application to synthetic data. Results obtained in Step 3 by using 
	$\sigma = 21$ (equation~\ref{eq:elements-wii}).
	The remaining informations are the same shown in the caption of
	Figure~\ref{fig:volc-margem-step1}.
	\label{fig:volc-margem-step3}}
}

\plot{volcanic-margin-grafics-estimated-model-alphas_X_1_1_2_more-known-depths}{width=0.6\textwidth}{
	{Application to synthetic data. Results obtained in Step 1 by using
	additional known depths at basement (purple symbols).
	The remaining informations are the same shown in the caption of
	Figure~\ref{fig:volc-margem-step1}.
	\label{fig:volc-margem-step1-more-inf}}
}

\plot{volcanic-margin-grafics-estimated-model-alphas_2_1_1_2-sgm_21_more-known-depths}{width=0.6\textwidth}{
	{Application to synthetic data. Results obtained in Step 3 by using 
	$\sigma = 21$ (equation~\ref{eq:elements-wii}) and 
	additional known depths at basement (purple symbols). 
	The remaining informations are the same shown in the caption of
	Figure~\ref{fig:volc-margem-step1}.
	Notice that, in comparison with the estimated model shown in 
	Figure~\ref{fig:volc-margem-step1-more-inf}, the number of additional
	known depths is small.
	\label{fig:volc-margem-step3-more-inf}}
}

\plot{pelotas-profile-grafics-estimated-model-alphas_2_1_1_2-sgm_58}{width=0.6\textwidth}{
	{Application to real data on Pelotas basin, Brazil. Results obtained in Step 3 by
	using $\sigma = 58$ (equation~\ref{eq:elements-wii}).
	(Bottom panel) Estimated surfaces, initial basement and Moho used in the
	inversion, as well as the known depths at basement and Moho.
	The dashed white lines represent an alternative interpretation obtained 
	by \citet{zalan2015} at the same study area.
	(Middle panel) Estimated lithostatic stress curve computed
	by using equation~\ref{eq:lithostatic-stress-densities}. The values are multiplied 
	by a constant gravity value equal to $9.81$ m/s$^{2}$.
	(Upper panel) Observed gravity disturbance data, data produced by the estimated
	model (predicted data) and data produced the by model used as initial guess in the
	inversion (initial guess data).
	The contour of the prisms forming the interpretation model were omitted.
	The density contrasts were defined according to Table~\ref{tab:volcanic-margin-model}.
	\label{fig:pelotas-step3}}
}

\plot{pelotas-profile-grafics-estimated-model-alphas_2_1_1_2-sgm_58_more-known-depths}{width=0.6\textwidth}{
	{Application to real data on Pelotas basin, Brazil. Results obtained in Step 3 by
	using $\sigma = 58$ (equation~\ref{eq:elements-wii}) and additional
	known depths at basement. The remaining informations are the same shown in
	the caption of Figure \ref{fig:pelotas-step3}.
	\label{fig:pelotas-step3-more-infs}}
}


%%%%%%%%%%%%%%%%%% Figures %%%%%%%%%%%%%%%%%%%%%%%%%%%%%%%%%%%%%%%%%%%%%%%

\section{ACKNOWLEDGMENTS}

\lipsum[0-1]


\newpage

\bibliographystyle{seg.bst}  % style file is seg.bst
\bibliography{bib-file.bib}


\end{document}

